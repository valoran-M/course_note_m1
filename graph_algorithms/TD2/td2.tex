\documentclass{article}

\usepackage[utf8]{inputenc}
\usepackage[T1]{fontenc}
\usepackage{amssymb}
\usepackage{ntheorem}
\usepackage{amsmath}
\usepackage{amssymb}
\usepackage[ a4paper, hmargin={3cm, 3cm}, vmargin={3cm, 3cm}]{geometry}

\usepackage{tcolorbox}
\tcbuselibrary{
  theorems,
  breakable,
  skins
}

\usepackage{hyperref}
\hypersetup{
    colorlinks,
    citecolor=black,
    filecolor=black,
    linkcolor=blue,
    urlcolor=blue
}

\newtcbtheorem[number within=section]
{correction}{Correction}{
  colback=white,
  colframe=black!70,
  separator sign none,
  description delimiters parenthesis,
  enhanced jigsaw,
  breakable,
  fonttitle=\bfseries
}{}

\theoremstyle{plain}
\theorembodyfont{\normalfont}
\theoremseparator{~--}
\newtheorem{exercice}{Exercice}[section]

\title{Graph Algorithms\\TD1 : Graph Colouring}
\author{Valeran MAYTIE}
\date{}

\begin{document}
  \maketitle

  \section{Some properties of colouring}

  \exercice What is the chromatic number of an even cycle $C_{2n}$ ? Of an odd
  cycle $C_{2n+1}$

  \exercice Show that a graph is bipartite if and only if it contains no odd
  cycle.

  \exercice Show that for every graph $G$, there exists an order on the vertices
  such that the greedy algorithm applied in this order returns a colouring with
  $\chi(G)$ colours.

  \exercice Prove that $\chi(G) \geq |V(G)| / \alpha(G)$, for every graph $G$.


  \section{Interval graphs}

  Given a set of intervals $\mathcal{I} = \{I_1, \ldots, I_n\}$ where $I_i =
  [a_i, b_i]$ for every $1 \geq i \geq n$, the interval graph associated whit
  $\mathcal I$ is the graph $G = (V, E)$ where $V = \{1, \ldots, n\}$ and $ij
  \in E$ iff $I_i$ and $I_j$ intersect, i.e. $a_i \leq b_j$ and $a_j \leq b_i$,
  for every $i \leq i, j \leq n$.

  \exercice Show that in an interval graph, there exists a simplicial vertex,
  i.e. a vertex $v$ such that $N[v]$ induces a clique.

  \exercice Write an algorithm that computes an optimal proper colouring of an
  interval graph $G$. You may assume that we know the intervals. The goal
  complexity is $\mathcal O(n\ln n + m)$.

  \exercice We now want to write an algorithms which computes a proper colouring
  of any graph $G$, and use $\chi(G)$ colours if $G$ is an interval graph (so in
  particular we don't know the intervals if this is the case). Show that this
  can be done with the greedy colouring algorithm applied with a reverse
  degeneracy ordering.
 
\end{document}
