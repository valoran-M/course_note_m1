\documentclass{article}

\usepackage[utf8]{inputenc}
\usepackage[T1]{fontenc}
\usepackage{amssymb}
\usepackage{ntheorem}
\usepackage{amsmath}
\usepackage{amssymb}
\usepackage[ a4paper, hmargin={3cm, 3cm}, vmargin={3cm, 3cm}]{geometry}

\usepackage{tcolorbox}
\tcbuselibrary{
  theorems,
  breakable,
  skins
}

\usepackage{hyperref}
\hypersetup{
    colorlinks,
    citecolor=black,
    filecolor=black,
    linkcolor=blue,
    urlcolor=blue
}

\newtcbtheorem[number within=section]
{correction}{Correction}{
  colback=white,
  colframe=black!70,
  separator sign none,
  description delimiters parenthesis,
  enhanced jigsaw,
  breakable,
  fonttitle=\bfseries
}{}

\theoremstyle{plain}
\theorembodyfont{\normalfont}
\theoremseparator{~--}
\newtheorem{exercice}{Exercice}[section]

\title{Graph Algorithms\\TD1 : Introduction}
\author{Valeran MAYTIE}
\date{}

\begin{document}
  \maketitle

  \section{To begin}
 
  \exercice Show that a graph always has an even number of odd degree vertices

  \begin{correction}{}{}
    Let $G$ a graph. Thanks to \textit{Handshaking Lemma} we have:

    $$ \sum_{v \in V(G)} deg(v) = 2|G(E)| $$

    So the number $\sum_{v \in V(G)} deg(v)$ is an even number. To keep this
    property we must have an even number of odd degree vertices otherwise the
    sum become odd.
  \end{correction}

  \exercice Show that a graph with at least 2 vertices contains 2 vertices of
    equal degree

  \begin{correction}{}{}
    Let $G$ a graph with at least 2 vertices.

    \begin{itemize}
      \item If $G$ has no isolated vertex, the degree of a vertex is
        between $1$ and $|V(G)| - 1$ ($v$ a vertex $1 \leq deg(v) < |V(G)|$).
        So we have $n := |V(G)| - 1$ different values for the degree of a vertex
        in $G$. The graph $G$ contains $n + 1$ vertices so by the Pigeonhole
        principle we have two vertices with the same degree.

      \item If $G$ has one isolated vertex, it's the same idea that the previous
        one but the degree of a vertex is between $0$ and $|V(G)| - 2$.

      \item If $G$ has at least two isolated vertices, we have $2$ vertices with
        the same degree.
    \end{itemize}

  \end{correction}

  \exercice Let $G$ be a graph of minimum degree $\delta(G) \leq 2$. Show that
  $G$ contains a cycle.

  \begin{correction}{}{}
    Let $G$ a graph with $\delta(G) \leq 2$.

    Show that $G$ has a cycle. Let $P = v_1 \ldots v_l$ be the maximum path in
    $G$. Since it cannot be extended into a larger path, we must have $N(v_l)
    \subseteq V(P)$. So $\exists i \leq l-1$ $v_i \in N(v_l)$ which yields that
    $v_i, \ldots , v_l$ is a cycle.
  \end{correction}

  \exercice Let $G$ be a graph of minimum degree $d$, and of girth $2t+1$. Given
  any vertex $v \in V(G)$, show that there are at least $d(d-1)^{i-1}$ vertices
  at distance exactly $i$ from $v$ in $G$, for every $1 \leq i \leq t$. Deduce a
  lower bound on the number of vertices of $G$.

  \begin{correction}{}{}
  \end{correction}


  \section{Dense subgraphs}

  \exercice Show that every graph of average degree $d$ contains a subgraph of
  minimum degree at least $\frac d 2$.

  \begin{correction}{}{}
    Let $G$ a graph with an average degree $d$. We take the maximum average
    degree subgraph of $G$ we note them $G'$.
  \end{correction}

  \exercice Can you find a similar relation between the maximum degree and the
  minimum degree ? And between the maximum degree and the average degree ?

  \exercice Show that every graph of average degree $d$ contains a bipartite
  subgraph of average degree at least $\frac d 2$.

  \section{Cuts and trees}

  \exercice If $G$ is connected , and $e=uv$ is a bridge in $G$, how many
  connected components does $G\backslash e$ contain ? Show that $u$ and $v$ are
  cut-vertices.

  \exercice Show that a graph $G$ is a tree if and only if there exists a unique
  path from $u$ to $v$ in $G$, for every pair of vertices $u, v \in G$.

  \exercice Let $T$ a BFS tree of a graph $G$. Show that every edge of $G$ is
  contained either within a layer of $T$, or between two consecutive layers of
  $T$.

  \exercice Let $T$ be a DFS tree of a graph $G$. Show that, for every edge $e
  \in E(G)$, ther is a branch of $T$ that contains both extremities of $e$.


\end{document}
