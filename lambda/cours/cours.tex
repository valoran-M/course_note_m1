\documentclass{article}

\usepackage[utf8]{inputenc}
\usepackage{enumitem}
\usepackage{multirow}
\usepackage{xcolor}
\usepackage[T1]{fontenc}
% \usepackage[french]{babel}
\usepackage{hyperref}
\usepackage{amssymb}
\usepackage{mathtools}
\usepackage{ntheorem}
\usepackage{amsmath}
\usepackage{amssymb}
\usepackage[ a4paper, hmargin={2cm, 2cm}, vmargin={3cm, 3cm}]{geometry}
\usepackage{capt-of}
\usepackage{multicol}

\usepackage[braket, qm]{qcircuit}
\usepackage{graphicx}

\usepackage{tikz}
\usetikzlibrary{angles,quotes}

\theoremstyle{plain}
\theorembodyfont{\normalfont}
\theoremseparator{~--}
\newtheorem{exo}{Exercise}%[section]

\newcommand{\norm}[1]{\left\lVert#1\right\rVert}

\usepackage{hyperref}
\hypersetup{
    colorlinks,
    citecolor=black,
    filecolor=black,
    linkcolor=blue,
    urlcolor=blue
}

\usepackage{xcolor}

\definecolor{codegreen}{rgb}{0,0.6,0}
\definecolor{codegray}{rgb}{0.5,0.5,0.5}
\definecolor{codepurple}{rgb}{0.58,0,0.82}

\usepackage{listings}
\lstdefinestyle{mystyle}{
    commentstyle=\color{codegreen},
    keywordstyle=\color{magenta},
    numberstyle=\tiny\color{codegray},
    stringstyle=\color{codepurple},
    basicstyle=\ttfamily\footnotesize,
    breakatwhitespace=false,
    breaklines=true,
    captionpos=b,
    keepspaces=true,
    numbers=left,
    numbersep=5pt,
    showspaces=false,
    showstringspaces=false,
    showtabs=false,
    tabsize=2
}
\lstset{style=mystyle}

\theoremstyle{plain}
\theorembodyfont{\normalfont}
\theoremseparator{~--}
\newtheorem*{proof}{Proof}
\newtheorem{exam}{Example}
\renewcommand\qedsymbol{$\square$}


\title{$\lambda$-calculus}
\author{Valeran MAYTIE}
\date{}

\begin{document}
  \maketitle

  \section{Presentation}

  \begin{itemize}
    \item 1935 (a theory of computable functions)

      Alonzo Church, attempt at formalizing computation
  \end{itemize}

  Functions:
  \begin{itemize}
    \item maths : $f : A \to B$ is a set of pairs
    \item programming : instruction to compute an output
  \end{itemize}

  \subsection{Definitions}

  We can define the set of $\lambda$-terms ($\Lambda$) with a grammar:
  \begin{align*}
    \Lambda :=&\; x, y, z ...         & (\text{variable}) \\
             |&\; \lambda. \Lambda    & (\text{functions}) \\
             |&\; \Lambda\; \Lambda   & (\text{application})
  \end{align*}

  The application is left associative: $(l_1\; l_2)\; l_3$.

  \subsection{Computation}

  Example, we want to compute $(\lambda x y z.\; x\; z\; (y\; z))\; (\lambda a
  b.\; a)\; t\; u$

  \begin{align*}
    &\;(\lambda x y z.\; x\; z\; (y\; z))\; (\lambda a b.\; a)\; t\; u \\
    =&\;(\lambda y z.\; (\lambda a b.\; a)\; z\; (y\; z))\; t\; u \\
    =&\;(\lambda z.\; (\lambda a b.\; a)\; z\; (t\; z))\; u \\
    =&\;(\lambda a b.\; a)\; u\; (t\; u))\\
    =&\;(\lambda b.\; u)\; (t\; u))\\
    =&\;u
  \end{align*}

  Here are some examples of slightly more subtle calculations:
  \begin{multicols}{2}
    \begin{align*}
      &(\lambda x.\; (\lambda x.\; x))\; y \\
      &=\; \lambda x.\; x
    \end{align*}

    \begin{align*}
      &(\lambda x.\; (\lambda y.\; x))\; y \\
      &=\; \lambda z.\; y
    \end{align*}
  \end{multicols}

  We will define the reduction rewrite rule called $\beta$-reduction later.

  \subsection{Inductive reasoning}

  We can also define $\Lambda$ with the smallest set such that :

  \begin{itemize}
    \item $\forall x \in \text{Var}, x \in \Lambda$
    \item $\forall x \in \text{Var}, \forall t \in \Lambda, \lambda x.t \in
      \Lambda$
    \item $\forall t_1 t_2, t_1\; t_2 \in \Lambda$
  \end{itemize}

  We define $\Lambda$ by induction, so we can write induction function.

  For example, we can write $f_v$ the function who compute the number of
  variable in term $t$ and $f_@$ the function who compute the number of
  application
  \begin{multicols}{2}
    \[
        \begin{cases}
            f_v(x) &= 1 \\
            f_v(\lambda x. t) &= f_v(t) \\
            f_v(t_1\; t_2) &= f_v(t_1) + f_v(t_2) \\
        \end{cases}
    \]

    \[
        \begin{cases}
            f_@(x) &= 0 \\
            f_@(\lambda x. t) &= f_@(t) \\
            f_@(t_1\; t_2) &= 1 + f_@(t_1) + f_@(t_2) \\
        \end{cases}
    \]
  \end{multicols}

  How to prove that some property $P(t)$ is valid for all $\lambda$-terms $t$ ?

  \begin{enumerate}
    \item Prove that $\forall x \in \text{Var}, P(x)$ is valid
    \item Prove that $\forall x \in \text{Var}, \forall t, P(t) \Rightarrow
      P(\lambda x.\; t)$ is valid
    \item Prove that $\forall t_1, t_2, P(t_1) \wedge P(t_2) \Rightarrow P(t_1
      \; t_2)$ is valid
  \end{enumerate}

  \exam We want to prove $H : \forall t, f_v(t) = 1 + f_@(t) $

  \proof We proof $H$ by induction on the term $t$ :

  \begin{itemize}
    \item $t = x$, $f_v(x) = x$ and $f_@(x) = 0$, so we have $f_v(x) = 1 +
      f_@(x)$

    \item $t = \lambda x. t$, we assume that $f_v(t) = 1 + f_@(t)$.
      We calculate $f_v(\lambda x. t) = f_v(t) = 1 + f_@(t) = 1 + f_@(\lambda
      x.t)$

    \item $t = t_1\; t_2$, we assume that $f_v(t_1) = 1 + f_@(t_1)$ and
      $f_v(t_2) = 1 + f_@(t_2)$. By the calculation $f_v(t_1\; t_2) = f_v(t_1) +
      f_v(t_2) = 1 + f_@(t_1) + 1 + f_@(t_2) = 1 + f_@(t_1\; t_2)$
  \end{itemize}
  \qedsymbol

  \subsection{Bound variables and free variables}

  \subsection{$\alpha$-equivalence}

  \subsection{$\beta$-reduction}

  \exam Make them nice
    \begin{itemize} 
      \item $\lambda x.\; (\lambda x.\; x\; y) (\lambda y. x\; y)$
      \item $\lambda x y.\; x (\lambda y.\; (\lambda y.\; y)\; y\; z)$
    \end{itemize}

  \exam Compute $(\lambda f.\; f\;f)\; (\lambda a b. b\;a\;b)$

  \exam Prove that $fv(t[x \leftarrow u]) \subseteq (fv(t) \ \{x\}) \cup fv(u)$

\end{document}
