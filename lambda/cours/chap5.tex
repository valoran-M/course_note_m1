\section{Typed $\lambda$-calculus}

  \subsection{Simply typed $\lambda$-calculus}

  The main objective is to create a type system for verified the
  well-formation of a $\lambda$-term.

  \begin{align*}
    \tau ::=&\; o & \text{base type} \\
        |&\; \tau \to \tau
  \end{align*}

  And now we define the Church-typed $\lambda$-terms.
  
  \begin{align*}
    t ::=&\; x \\
      |&\; \lambda x^\sigma. t \\
      |&\; t\;t
  \end{align*}

  We define the typing rule with this formula :

  $$\Gamma \vdash t : \tau$$

  The $\Gamma$ is the context (environment) $\Gamma = \{x_1^{\sigma_1}, \ldots,
  x_n^{\sigma_n}\}$.

  We can type a $\lambda$-term with the following inference rules :

  \begin{mathpar}
    \inferrule*[Right=Var]{x^\sigma \in \Gamma}{\Gamma \vdash x : \sigma} \\

    \inferrule*[Left=Abs]{\Gamma, x^\sigma \vdash t : \tau}
              {\Gamma \vdash \lambda x. t : \sigma \to \tau} \and
    \inferrule*[Left=App]{\Gamma \vdash t : \sigma \to \tau \\ 
      \Gamma \vdash u : \sigma}
              {\Gamma \vdash t\; u : \tau}
  \end{mathpar}

  For example, we have this derivation tree :

  \begin{mathpar}
    \inferrule*[Right=Abs]
      {\inferrule*[Right=Var]{x \in \{x^\sigma\}}{x^\sigma \vdash x : \sigma}}
      {\vdash \lambda x. x : \sigma \to \sigma} \\

    \inferrule*[Right=3$\times$Abs]
      { 
      \inferrule*[Right=App]
      { 
        \inferrule*[Left=Var]{
        }
        {x^{\beta \to \gamma}, y^{\alpha \to \beta, z^\alpha} x : \beta \to
          \gamma} \\
        \inferrule*[Right=App]{ 
          \inferrule*[Left=Var]{ }
            {x^{\beta \to \gamma}, y^{\alpha \to \beta, z^\alpha} y : \alpha \to
          \beta} \\
          \inferrule*[Right=Var]{ }
            {x^{\beta \to \gamma}, y^{\alpha \to \beta, z^\alpha} z : \alpha}
        }
        {x^{\beta \to \gamma}, y^{\alpha \to \beta, z^\alpha} y\;z : \beta} \\
      }
      {x^{\beta \to \gamma}, y^{\alpha \to \beta, z^\alpha} x\;(y\;z) : \gamma}
      }
      {\vdash \lambda x\;y\;z. x\;(y\;z) : (\beta \to \gamma) \to (\alpha \to
          \beta) \to \alpha \to \gamma} \\

    \inferrule*[Right=Abs]
      { 
        \inferrule*[Right=App]
        {
          \inferrule*[Left=Var]{\color{red}\text{No solution}}
          {x^\sigma \vdash x : \sigma \to \tau} \\
          \inferrule*[Right=Var]{ }
          {x^\sigma \vdash x : \sigma}
        }
        {x^\sigma \vdash x\;x : \tau}
      }
      {\vdash \lambda x. x\;x : \sigma \to \tau}
  \end{mathpar}
  
  \lemma Type preservation by substitution

  $$\Gamma, x^\sigma \vdash t : \tau \wedge \Gamma \vdash u : \tau
    \Rightarrow \Gamma \vdash t\{x \leftarrow u\} : \tau
  $$

  \textit{Proof}: By induction on the derivation of
  $\Gamma, x^\sigma \vdash t : \tau$

  \begin{itemize}
    \item Case $\Gamma, x^\sigma \vdash y : \tau$
      \begin{itemize}
        \item if $y \not = x$ then $y\{x \leftarrow u\} = y$
          we have already $\Gamma, x^\sigma \vdash y : \tau$
        \item if $y = x$ then $y\{x \leftarrow u\} = u$
          one again we have $\Gamma\vdash u : \tau$ in our hypothesis.
      \end{itemize}

    \item Case $\Gamma, x^\sigma \vdash \lambda y.v : \tau_1 \to \tau_2$ with
      $\Gamma, x^\sigma, y^{\tau_1} \vdash v : \tau_2$

      By induction hypothesis we have $\Gamma, x^\sigma
      \vdash v\{x\leftarrow u\} : \tau_2$

    \item Case $\Gamma, x^\sigma \vdash t_1\;t_2 : \tau$ with 
      $\Gamma, x^\sigma \vdash t_1 : \tau ' \to \tau$ and
      $\Gamma, x^\sigma \vdash t_2 : \tau '$.
      
      By induction hypothesis we have
      $\Gamma, x^\sigma \vdash t_1\{x \leftarrow u\} : \tau ' \to \tau$ and
      $\Gamma, x^\sigma \vdash t_2\{x\leftarrow\} : \tau '$.
  \end{itemize}

  \qedsymbol

  \lemma Type preservation lemma (subject reduction lemma)

  $$\Gamma\vdash t : \tau \wedge t \to_\beta \Rightarrow \Gamma \vdash t' : \tau$$

  \textit{Proof}: By induction on $\Gamma \vdash t : \tau$

  \begin{itemize}
    \item Case $\Gamma \vdash x : \sigma$ with $x^\sigma \in \Gamma$
      
      We can not reduce $x$, so we don't have $t'$ such as $t\rbeta t'.$

    \item Case $\Gamma \vdash \lambda x.t$ with $\Gamma, x^\sigma \vdash t :
      \tau$. Assume $\lambda x. t \rbeta \lambda x. t'$ with $t \rbeta t'$.

      By induction hypothesis we have $\Gamma, x^\sigma t' : \tau$. The by the
      rule \textit{Abs} we have $\Gamma \vdash \lambda x. t' : \sigma : \tau$

    \item Case $\Gamma \vdash t\;u: \tau$ with $\Gamma t : \sigma \to \tau$ and
      $\Gamma \vdash u : \sigma$.

      We have 3 case on $\rbeta$:

      \begin{itemize}
        \item $t\;u \rbeta t'\;u$ with $t \rbeta t'$ 
          By induction hypothesis we have $\Gamma \vdash t' : \sigma \to \tau$

          The by \textit{App} we deduce $\Gamma \vdash t'\;u: \tau$.

        \item $t\;u \rbeta t\;u'$ with $u \rbeta  u'$ (Similar)

        \item $(\lambda x.v)\; u \rbeta v\{x \leftarrow u\}$
      \end{itemize}
  \end{itemize}
  \qedsymbol

  \subsection{Strong normalization property}

  $(E, \leq)$ such that $\leq$ is an order relation (reflexive, transitive and
  anti-symmetric).\\
  Well-founded order : order with no infinite strictly decreasing sequence. \\
  (WF) Well-founded induction principle :

  $P$ : property on the elements of $E$,
  $\forall x, (\forall y, y < x \Rightarrow P(x)) \Rightarrow \forall P (x)$

  How to use it ?

  Let $x \in E$, Assume $P(y)$ for all $y < x$ Prove $P(x)$

  \lemma
  If $\Gamma \vdash t' : \tau'$ and $\Gamma, x^{\tau'} \vdash t : \tau$ then $t\{x
  \leftarrow t'\}$ is SN.

  \textit{Proof}:
  
  We will use the well-founded induction principle on the triplet $(|t'|)$ with the
  lexical order $\leq_l$.

  \qedsymbol

  \lemma
  If $\Gamma \vdash t : \tau$ then $t$ is a SN (no infinite reduction sequence
  from t).

  \textit{Proof}:

  By induction on $t$.

  \begin{itemize}
    \item If $t = x$, then $t$ is already in normal form, so it is SN.

    \item If $t = \lambda x. t_0$. Since $t$ is typed, then $t_0$ is also typed
      (rule \textit{Abs}). So by the induction hypothesis we know that $t_0$ is
      strongly normalizable. So $t$ is strongly normalizable.

    \item If $t = t_0\; t_1$, as before $t_0$ and $t_1$ are typed, so they are
      also strongly normalizable.

      If $t_0$ is never reduce to a $\lambda$-term of the form $\lambda y. t'$,
      then $t_0\;t_1$ is strongly normalizable by induction hypothesis.

      Otherwise, we end up with $t_0\;t_1 \rsbeta (\lambda y.t')\;t_1' \rbeta
      t'\{x \leftarrow t_1'\}$ such
      that $t_1 \rsbeta t_1'$, $t_0 \rsbeta$, $\Gamma \vdash t_1' : \tau_1'$ and
      $\Gamma, y^{\tau_1} \vdash t' : \tau'$.
  \end{itemize}


  \qedsymbol



