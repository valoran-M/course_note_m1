\section{Computing with the $\lambda$-calculus}

  Example, we want to compute $(\lambda x y z.\; x\; z\; (y\; z))\; (\lambda a
  b.\; a)\; t\; u$

  \begin{align*}
    &\;(\lambda x y z.\; x\; z\; (y\; z))\; (\lambda a b.\; a)\; t\; u \\
    =&\;(\lambda y z.\; (\lambda a b.\; a)\; z\; (y\; z))\; t\; u \\
    =&\;(\lambda z.\; (\lambda a b.\; a)\; z\; (t\; z))\; u \\
    =&\;(\lambda a b.\; a)\; u\; (t\; u))\\
    =&\;(\lambda b.\; u)\; (t\; u))\\
    =&\;u
  \end{align*}

  Here are some examples of slightly more subtle calculations:
  \begin{multicols}{2}
    \begin{align*}
      &(\lambda x.\; (\lambda x.\; x))\; y \\
      &=\; \lambda x.\; x
    \end{align*}

    \begin{align*}
      &(\lambda x.\; (\lambda y.\; x))\; y \\
      &=\; \lambda z.\; y
    \end{align*}
  \end{multicols}

  We will define the reduction rewrite rule called $\beta$-reduction later.

  \subsection{Inductive reasoning}

  We can also define $\Lambda$ with the smallest set such that :

  \begin{itemize}
    \item $\forall x \in \text{Var}, x \in \Lambda$
    \item $\forall x \in \text{Var}, \forall t \in \Lambda, \lambda x.t \in
      \Lambda$
    \item $\forall t_1 t_2, t_1\; t_2 \in \Lambda$
  \end{itemize}

  We define $\Lambda$ by induction, so we can write induction function.

  For example, we can write $f_v$ the function who compute the number of
  variable in term $t$ and $f_@$ the function who compute the number of
  application
  \begin{multicols}{2}
    \[
        \begin{cases}
            f_v(x) &= 1 \\
            f_v(\lambda x. t) &= f_v(t) \\
            f_v(t_1\; t_2) &= f_v(t_1) + f_v(t_2) \\
        \end{cases}
    \]

    \[
        \begin{cases}
            f_@(x) &= 0 \\
            f_@(\lambda x. t) &= f_@(t) \\
            f_@(t_1\; t_2) &= 1 + f_@(t_1) + f_@(t_2) \\
        \end{cases}
    \]
  \end{multicols}

  How to prove that some property $P(t)$ is valid for all $\lambda$-terms $t$ ?

  \begin{enumerate}
    \item Prove that $\forall x \in \text{Var}, P(x)$ is valid
    \item Prove that $\forall x \in \text{Var}, \forall t, P(t) \Rightarrow
      P(\lambda x.\; t)$ is valid
    \item Prove that $\forall t_1, t_2, P(t_1) \wedge P(t_2) \Rightarrow P(t_1
      \; t_2)$ is valid
  \end{enumerate}

  \exam We want to prove $H : \forall t, f_v(t) = 1 + f_@(t) $

  \proof We proof $H$ by induction on the term $t$ :

  \begin{itemize}
    \item $t = x$, $f_v(x) = x$ and $f_@(x) = 0$, so we have $f_v(x) = 1 +
      f_@(x)$

    \item $t = \lambda x. t$, we assume that $f_v(t) = 1 + f_@(t)$.
      We calculate $f_v(\lambda x. t) = f_v(t) = 1 + f_@(t) = 1 + f_@(\lambda
      x.t)$

    \item $t = t_1\; t_2$, we assume that $f_v(t_1) = 1 + f_@(t_1)$ and
      $f_v(t_2) = 1 + f_@(t_2)$. By the calculation $f_v(t_1\; t_2) = f_v(t_1) +
      f_v(t_2) = 1 + f_@(t_1) + 1 + f_@(t_2) = 1 + f_@(t_1\; t_2)$
  \end{itemize}
  \qedsymbol

  \subsection{Variables and substitutions}

  \subsubsection{Free and bound variables}

  To define more calculation operations, we define free variables and bound
  variables.

  Informally, free variables are variables used, but linked to no lambda
  abstraction. While linked variables are those used and linked to a lambda
  abstraction.

  Definition :

  \begin{multicols}{2}
    \[
      \begin{cases}
        fv(x) &= \{x\} \\
        fv(\lambda x. t) &= fv(t)\backslash \{x\} \\
        fv(u\; v) &= fv(u) \cup fv(v) \\
      \end{cases}
    \]

    \[
      \begin{cases}
        bv(x) &= \emptyset \\
        bv(\lambda x. t) &= \{x\} \cup bv(t) \\
        bv(u\; v) &= bv(u) \cup bv(v) \\
      \end{cases}
    \]
  \end{multicols}

  \subsubsection{Substitution}

  The substitution is an operation on $\lambda$-term. The aim is to replace the
  free occurrences of a variable $x$ in term $t$ with another $\lambda$-term
  $u$. It is noted : $t\{x \leftarrow u\}$. We can define this operation by
  induction on a $\lambda$-term :

  \begin{align*}
    y\{x \leftarrow u\} &= \begin{cases}
      u & \text{if } x = y \\
      y & \text{if } x \not = y
    \end{cases} \\
    (t_1\; t_2)\{x \leftarrow u\} &= t_1\{x \leftarrow u\}\; t_2\{x \leftarrow
    u\} \\
    (\lambda y.\;t)\{x \leftarrow u\} &= \begin{cases}
      \lambda y.\;t &\text{if } x = y \\
      \lambda y.\;t\{x\leftarrow u\} &\text{if } x \not = y \text{ and } y \not
      \in fv(u)\\
      \lambda z.\;t\{y\leftarrow z\}\{x\leftarrow u\} &\text{if } x \not = y \text{ and } y \in
      fv(u) \text{\hspace{1cm}$z$ fresh}\\
    \end{cases}
  \end{align*}

  \paragraph{Barendregt's convention} The definition of substitution above is
  not very easy to handle. So we are going to use a convention to greatly
  simplify the substitution :

  \begin{center}
    \textit{no variable name appears both free and bound in any given subterm}
  \end{center}

  \begin{center}
    \begin{tabular}{c|c}
      Not Good & Good \\
      \hline
      $\lambda x.\;x\;(\lambda x.\;x)$ & $\lambda x.\;(x\;(\lambda y.\;y))$
    \end{tabular}
  \end{center}

  The substitution definition become :

  \begin{align*}
    y\{x \leftarrow u\} &= \begin{cases}
      u & \text{if } x = y \\
      y & \text{if } x \not = y
    \end{cases} \\
    (t_1\; t_2)\{x \leftarrow u\} &= t_1\{x \leftarrow u\}\; t_2\{x \leftarrow
    u\} \\
    (\lambda y.\;t)\{x \leftarrow u\} &= \lambda y.\; t\{x \leftarrow u\}
  \end{align*}

  \paragraph{(Un)stability of Barendregt's convention}
  Sometimes during the computation we need to change variables name to preserve
  the convention :

  \begin{align*}
    &(\lambda x.\;x\;x)\;(\lambda yz.\;y\;z)\\
    &\to (\lambda yz.\;y\;z)\;(\lambda yz.\;y\;z)\\
    &\to (\lambda yz.\;(\lambda yz.\;y\;z)\;z) & \text{Wrong}\\
  \end{align*}

  \subsubsection{$\alpha$-conversion}

  Two term can be structurally different, but with the same meaning ($\lambda
  x.\; x$ and $\lambda y.\; y$). We can therefore rename linked variables under
  certain conditions without changing the meaning of a lambda term. We call this
  operation $\alpha$-conversion or $\alpha$-renaming.

  $\alpha$-conversion definition :

  \begin{align*}
    \lambda x.\; t &=_\alpha \lambda y.\;{x \leftarrow y} & \text{with } x,y
      \not \in bd(t) \text{ and } y \not \in fv(t)
  \end{align*}

  The $\alpha$-conversion is a congruence :

  \begin{align*}
    t =_\alpha t' &\Rightarrow \lambda x.\;t =_\alpha t' \\
    t_1 =_\alpha t_1' &\Rightarrow t_1\;t_2 =_\alpha t_1'\;t_2 \\
    t_2 =_\alpha t_2' &\Rightarrow t_1\;t_2 =_\alpha t_1\;t_2' \\
  \end{align*}

  From now on we assume that any term we work with satisfies Barendregt’s
  convention.

  \exo Make them nice
    \begin{itemize}
      \item $\lambda x.\; (\lambda x.\; x\; y) (\lambda y. x\; y)$
      \item $\lambda x y.\; x (\lambda y.\; (\lambda y.\; y)\; y\; z)$
    \end{itemize}

  \textit{Answer :}
    \begin{itemize}
      \item $\lambda x.\; (\lambda x.\; x\; y) (\lambda y. x\; y) =_\alpha
        \lambda x.\; (\lambda z.\; z\; y) (\lambda w. x\; w)$
      \item $\lambda x y.\; x (\lambda y.\; (\lambda y.\; y)\; y\; z) =_\alpha
        \lambda x y.\; x (\lambda a.\; (\lambda t.\; t)\; a\; z)$
    \end{itemize}

  \exo Compute $(\lambda f.\; f\;f)\; (\lambda a b. b\;a\;b)$

  \textit{Answer :}
    \begin{align*}
      (\lambda f.\;f\;f)\; (\lambda a\; b.\;b\;a\;b) &\to_\beta
      (\lambda a b.\;b\;a\;b)\; (\lambda a\;b.\;b\;a\;b) \\
      &\to_\beta \lambda b.\;b\;(\lambda a\;b.\;b\;a\;b)\;b\\
      &=_\alpha \lambda b.\;b\;(\lambda x\;y.\;y\;x\;y)\;b\\
      &\to_\beta \lambda b.\;b\;(\lambda y.\;y\;b\;y)\\
    \end{align*}


  \exo Prove that $fv(t[x \leftarrow u]) \subseteq (fv(t) \backslash \{x\})
    \cup fv(u)$

  \textit{Answer :} Proof by induction on the structure of $t$

  \begin{itemize}
    \item Case where $t$ is a variable
      \begin{itemize}
        \item case x: $fv(x\{x\leftarrow u\}) = fv(u) \subseteq (fv(t)
          \backslash \{x\} \cup fv(u)$
        \item case $y\not =x$: $fv(y\{x\leftarrow u\} ) fv(y) = \{y\}$ and
          $\{y\}$ is indeed a subset of ($fv(y)\backslash \{x\}) \cup fv(u) =
          \{y\} \cup fv(u)$)
      \end{itemize}

    \item case where $t$ is an application $t_1\;t_2$. Assume
      $fv(t_1\{x\leftarrow y\})\subseteq fv(t_1) \backslash \{x\} \cup fv(u)$ and
      $fv(t_2\{x\leftarrow y\})\subseteq fv(t_2) \backslash \{x\} \cup fv(u)$.
      Then

      \begin{align*}
        &fv((t_1\;t_2)\{x \leftarrow u\}) \\
        &= fv(t_1\{x \leftarrow u\}\; t_2\{x \leftarrow u\}) & \text{by
        definition of the substitution} \\
        &= fv(t_1\{x \leftarrow u\}) \cup fv(t_2)\{x \leftarrow u\}) & \text{by
        definition of $fv$} \\
        &\subseteq fv(t_1 \backslash \{x\} \cup fv(u) \cup
        fv(t_2 \backslash \{x\} \cup fv(u) & \text{by induction hypothesis} \\
        &=fv(t_1 \backslash \{x\} \cup fv(t_2) \backslash \{x\}
        \cup fv(u) \\
        &=(fv(t_1 \cup fv(t_2)) \backslash \{x\}
        \cup fv(u) \\
        &= (fv(t_1\;t_2) \backslash \{x\}) \cup fv(u)
      \end{align*}

    \item Case where $t$ is $\lambda$-abstraction $\lambda y.t_0$. Asume
      $fv(t_0\{x\leftarrow u\} \subseteq (fv(t_0) \backslash \{x\} \cup fv(u)$.
      Then

      \begin{align*}
        &fv(\lambda y.t_0)\{x \leftarrow u\} \\
        &= fv(\lambda y.t_0\{x \leftarrow u\}) \\
        &= fv(t_0\{x \leftarrow u\})\{y\} \\
        &\subseteq ((fv(t_0) \backslash\{x\})\cup fv(u))\backslash \{y\} &
        \text{induction hypothesis} \\
        &= (fv(t_0) \backslash\{x\} \backslash \{y\})\cup fv(u)\backslash \{y\}
        \\
        &= (fv(t_0) \backslash\{x\} \backslash \{y\})\cup fv(u) \\
        &= (fv(t_0) \backslash \{y\} \backslash\{x\})\cup fv(u) \\
        &= (fv(\lambda y.t_0) \backslash \{x\} \cup fv(u) \\
      \end{align*}
  \end{itemize}
  \qedsymbol

  \exo \label{exo:commsubst} Prove when $x\not\in fv(v)$ and $x \not = y$ then
  $t\{x\leftarrow u\}\{y\leftarrow v\} = t\{y\leftarrow v\}\{x\leftarrow u\{y\leftarrow v\}\}$

  \textit{Answer} Proof by induction on $t$
  \begin{itemize}
    \item Case where $t$ is a variable $z$:
      \begin{itemize}
        \item $z = x$ :
          \begin{multicols}{2}
            \begin{align*}
              &x\{x\leftarrow u\}\{y\leftarrow v\} \\
              &= u\{y\leftarrow v\} \\
            \end{align*}

            \begin{align*}
              &x\{y\leftarrow v\}\{x\leftarrow u\{y\leftarrow v\}\} \\
              &= x\{x\leftarrow u\}\{y\leftarrow v\}\} \\
              &= u\{y\leftarrow v\} \\
            \end{align*}
          \end{multicols}

        \item $z = y$ :
          \begin{multicols}{2}
            \begin{align*}
              &y\{x\leftarrow u\}\{y\leftarrow v\} \\
              &= y\{y\leftarrow v\} \\
              &= v \\
            \end{align*}

            \begin{align*}
              &y\{y\leftarrow v\}\{x\leftarrow u\{y\leftarrow v\}\} \\
              &= v\{x\leftarrow u\{y\leftarrow v\}\} \\
              &= v & \text{$x$ is not free in $v$} \\
            \end{align*}
          \end{multicols}
        \item $z \not = y$ and $z \not = x$ :
          $z\{x\leftarrow u\}\{y\leftarrow v\} = z$ and
          $z\{y\leftarrow v\}\{x\leftarrow u\{y\leftarrow v\}\} = z$
      \end{itemize}

    \item $t = t_1\; t_2$:
      \begin{align*}
        &(t_1\;t_2)\{x\leftarrow u\}\{y\leftarrow v\} \\
        &t_1\{x\leftarrow u\}\{y\leftarrow v\}\;t_2\{x\leftarrow u\}\{y\leftarrow v\} \\
        &t_1\{y\leftarrow v\}\{x\leftarrow u\{y\leftarrow v\}\}\;
         t_2\{y\leftarrow v\}\{x\leftarrow u\{y\leftarrow v\}\} &
         \text{induction hypothesis} \\
        &(t_1\;t_2)\{y\leftarrow v\}\{x\leftarrow u\{y\leftarrow v\}\}\\
      \end{align*}

    \item $t = \lambda z.\;t_0$
      \begin{align*}
        &(\lambda z.\;t_0)\{x\leftarrow u\}\{y\leftarrow v\} \\
        &\lambda z.\;t_0\{x\leftarrow u\}\{y\leftarrow v\} \\
        &\lambda z.\;t_0\{y\leftarrow v\}\{x\leftarrow u\{y\leftarrow v\}\} &
        \text{induction hypothesis}\\
        &(\lambda z.\;t_0)\{y\leftarrow v\}\{x\leftarrow u\{y\leftarrow v\}\} \\
      \end{align*}

  \end{itemize}
  \qedsymbol

  \subsection{$\beta$-reduction}

    The $\beta$-reduction is a rewrite rule who apply an argument to a function.
    We need to have a $\lambda$-term on the form $(\lambda x.\;t)\; u$. This
    form is called a ($\beta$-redex). The computation rule is :
    \[(\lambda x.\;t)\; u \to_\beta t\{x \leftarrow u\}\]

    We can draw :

    \begin{center}
    \begin{tikzpicture}[line width=0.3mm]
      \node(App1) at (0, 0) {$@$};
        \node(Abs1) at (-2, -1) {$\lambda x$};
        \node(Abs10) at (-2, -2) {$\lambda y$};
        \node(App100) at (-2, -3) {$@$};
          \node(App1001) at (-3, -4) {$@$};
            \node(Var10011) at (-3.5, -5) {$x$};
            \node(Var10012) at (-2.5, -5) {$y$};
          \node(App1002) at (-1, -4) {$@$};
            \node(Var10021) at (-1.5, -5) {$x$};
            \node(Abs10022) at (-0.5, -5) {$\lambda z$};
            \node(Var100220) at (-0.5, -6) {$z$};

        \node(Abs2) at (2, -1) {$\lambda a$};
        \node(Abs20) at (2, -2) {$\lambda b$};
        \node(App200) at (2, -3) {$@$};
          \node(Var2001) at (1.5, -4) {$b$};
          \node(Var2002) at (2.5, -4) {$a$};

      \draw (App1) -- (Abs1) (App1) -- (Abs2)
            (Abs1) -- (Abs10) (Abs10) -- (App100) -- (App1001) -- (Var10011)
                                                     (App1001) -- (Var10012)
            (App100) -- (App1002) -- (Var10021)
                        (App1002) -- (Abs10022) -- (Var100220)
            (Abs2) -- (Abs20) -- (App200) -- (Var2001) (App200) -- (Var2002);
      \node at (4, -2.5) {$\rightarrow_\beta$};

      \node(Abs0) at (7, 0) {$\lambda y$};
      \node(App00) at (7, -1) {$@$};
        \node(App001) at (6, -2) {$@$};
          \node(Abs0011) at (5.5, -3) {$\lambda a$};
          \node(Abs00110) at (5.5, -4) {$\lambda b$};
          \node(App001100) at (5.5, -5) {$@$};
            \node(Var0011001) at (5, -6) {$b$};
            \node(Var0011002) at (6, -6) {$a$};

          \node(Var0012) at (6.5, -3) {$y$};
        \node(App002) at (8, -2) {$@$};
          \node(Abs0021) at (7.5, -3) {$\lambda a$};
          \node(Abs00210) at (7.5, -4) {$\lambda b$};
          \node(App002101) at (7.5, -5) {$@$};
            \node(Var0021011) at (7, -6) {$b$};
            \node(Var0021012) at (8, -6) {$a$};
          \node(Abs0022) at (8.5, -3) {$\lambda z$};
          \node(Var00220) at (8.5, -4) {$z$};

      \draw (Abs0) -- (App00) -- (App001) -- (Abs0011) -- (Abs00110) -- (App001100) -- (Var0011001)
                                                                        (App001100) -- (Var0011002)
                                 (App001) -- (Var0012)
                      (App00) -- (App002) -- (Abs0021) -- (Abs00210) -- (App002101) -- (Var0021011)
                                                                        (App002101) -- (Var0021012)
                                 (App002) -- (Abs0022) -- (Var00220);
    \end{tikzpicture}
    \end{center}

  A $\beta$-reduction can be done anywhere in a term. We must therefore manage
  cases where a reduction is made after a lambda abstraction or in the left or
  right branch of an application. So we're going to describe our reduction
  using inference rule :


  \begin{mathpar}
    \inferrule { }
      {(\lambda x.\;t)\; u \\ \to_\beta \\ t\{x \leftarrow u\}} \\
    \inferrule
      {t \\ \to_\beta \\ t'}
      {t\;u \\ \to_\beta \\ t'\; u} \and
    \inferrule
      {u \\ \to_\beta \\ u'}
      {t\;u \\ \to_\beta \\ t\; u'} \\
    \inferrule
      {t \\ \to_\beta \\ u'}
      {\lambda x.\;t \\ \to_\beta \\ \lambda x.\;t'}
  \end{mathpar}


  \subsubsection{Position}

  We can locate the beta reduction by encoding the position of the reduction
  operation, we can rewrite the resets like this :

  \begin{mathpar}
    \inferrule { }
      {(\lambda x.\;t)\; u \\ \xrightarrow{\epsilon}_\beta \\ t\{x \leftarrow u\}} \\
    \inferrule
      {t \\ \xrightarrow{p}_\beta \\ t'}
      {t\;u \\ \xrightarrow{1 \cdot p}_\beta \\ t'\; u} \and
    \inferrule
      {u \\ \xrightarrow{p}_\beta \\ u'}
      {t\;u \\ \xrightarrow{2 \cdot p}_\beta \\ t\; u'} \\
    \inferrule
      {t \\ \xrightarrow{p}_\beta \\ u'}
      {\lambda x.\;t \\ \xrightarrow{0 \cdot p}_\beta \\ \lambda x.\;t'}
  \end{mathpar}

  \subsubsection{Inductive reasoning on reduction}

    Since the $\beta$-reduction has been defined using inference rules, we can
    resonate by recurrence on the reduction. To prove a formula of the form :

    \[
      \forall t, t', t \to_\beta t' \Rightarrow P(t, t')
    \]

    we need to check the following four points:

    \begin{align*}
      &\bullet P((\lambda x.\;t)u, t\{x\rightarrow u\}) \text{ for any } x, y \text{ and
      } u \\
      &\bullet P(t\;u, t'\;u) \text{for any } t, t' \text{ and } u \text{ such that }
      P(t, t') \\
      &\bullet P(t\;u, t\;u') \text{for any } t, u \text{ and } u' \text{ such that }
      P(u, u') \\
      &\bullet P(\lambda x.\;t, \lambda x.\;t') \text{for any } x, t \text{ and } u' \text{ such that }
      P(t, t') \\
    \end{align*}

  \exam We want to prove :

    \[\forall t\;t', t \to t' \Rightarrow fv(t') \subseteq fv(t)\]

  Proof by induction on the derivation of $t \to t'$.

  \begin{itemize}
    \item $(\lambda x.\;t)\;u \to t\{x \leftarrow u\}$. We already proved:
      $fv(t\{x \leftarrow y\} \subseteq (fv(t) \backslash \{x\} \cup fv(u)$.
      Moreover, we have

      \begin{align*}
        fv((\lambda x.\;t)\;u) &= fv(\lambda x.\;t) \cup fv(u) \\
            &= (fv(t)\backslash \{x\}) \cup fv(u)
      \end{align*}
    \item $t\;u t'\; u$ with $t \to t'$. Then

      \begin{align*}
        fv(t'\;u) &= fv(t') \cup fv(u) & \text{by definition} \\
        &\subseteq fv(t) fv(u) & \text{by induction hypothesis} \\
        &= fv(t\; u) & \text{by definition}
      \end{align*}

    \item $t\; u' \to t\; u'$ with $u \to u'$ similar.

    \item $\lambda x.\;t \to \lambda x.\; t'$ with $t\to t'$. Then

      \begin{align*}
        fv(\lambda x.\; t') &= fv(t')\backslash\{x\} & \text{by definition} \\
        &\subseteq fv(t) \backslash \{x\} & \text{by induction hypothesis} \\
        &= fv(\lambda x.\; t) & \text{by definition}
      \end{align*}
  \end{itemize}
  \qedsymbol

  \subsubsection{Reduction sequences}

  Other reduction can be set above the beta reduction :

  \begin{itemize}
    \item $\to_\beta$ one step
    \item $\to_\beta^* / \toto_\beta$ reflexive transitive closure: 0, 1 or many steps
    \item $\leftrightarrow$ symmetric closure : one step, forward or backward.
    \item $=_\beta$ reflexive, symmetric, transitive closure (equivalence)
  \end{itemize}

  \subsubsection{Context}

  We can also formalize our reduction with a context that is intuitively a
  lambda term with a hole. So we have two steps, replace a variable with a hole
  and replace the hole with a lambda term. So there is no need for substitution.

  We define a context with the following grammar:

  \begin{align*}
    \mathcal C :=&\; \square              & (\text{hole})\\
             |&\; x, y, z ...             & (\text{variable}) \\
             |&\; \lambda x.\; \mathcal C & (\text{functions}) \\
             |&\; \mathcal C\; \mathcal C & (\text{application})
  \end{align*}

  The operation $\mathcal C[u]$ is the result of filling the hole of $\mathcal
  C$ with the term $u$.

  \exo Here are some decompositions of $\lambda x.(x\; \lambda y.xy)$ into a
  context and a term $\mathcal C[u]$

  \begin{center}
  \begin{tabular}{c|c|c|c|c|c}
    $\mathcal C$ & $\square$ & $\lambda x.\square$ & $\lambda x.(\square\;
    \lambda y.xy)$ & $\lambda x.(x\; \square)$ & $\ldots$ \\
    \hline
    $u$ & $\lambda x.(x\; \lambda y.xy)$ & $x\; \lambda y.xy$ & $x$ & $\lambda
    y.\; x\;y$ & $\ldots$
  \end{tabular}
  \end{center}

  What are the other possible decompositions ?

  We already showed that

  \[
    (\lambda x.x\; ((\lambda y.zy)x))z \to (\lambda x.x\; (zx))\; z
  \]
  What are the context and the redex associated to this reduction ?

  \textit{Answer} Other decompositions of $\lambda x.(x\; \lambda y.xy)$

  \begin{center}
  \begin{tabular}{c|c|c|c|}
    $\mathcal C$ & $\lambda x.(x\; \lambda y.\square)$ & $\lambda x.(x\; \lambda
    y.\square\; y)$ & $\lambda x.(x\; \lambda y. x\; \square)$ \\
    \hline
    $u$ & $xy$ & $x$ & $y$
  \end{tabular}
  \end{center}

  Decomposition of the reduction:

  \[
    \mathcal C[(\lambda y.xy)x] \to \mathcal C [zx]
  \]

  with $\mathcal C = (\lambda x.x\;\square)\;z$


