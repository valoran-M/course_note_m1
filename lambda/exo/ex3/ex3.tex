\documentclass{article}

\usepackage[utf8]{inputenc}
\usepackage{enumitem}
\usepackage{multirow}
\usepackage{xcolor}
\usepackage[T1]{fontenc}
% \usepackage[french]{babel}
\usepackage{hyperref}
\usepackage{amssymb}
\usepackage{mathtools}
\usepackage{ntheorem}
\usepackage{amsmath}
\usepackage{amssymb}
\usepackage[ a4paper, hmargin={2cm, 2cm}, vmargin={3cm, 3cm}]{geometry}
\usepackage{capt-of}
\usepackage{multicol}
\usepackage{mathpartir}

\usepackage[braket, qm]{qcircuit}
\usepackage{graphicx}

\usepackage{tikz}
\usetikzlibrary{angles,quotes, 3d}

\usepackage{hyperref}
\hypersetup{
    colorlinks,
    citecolor=black,
    filecolor=black,
    linkcolor=blue,
    urlcolor=blue
}

\usepackage{xcolor}

\definecolor{codegreen}{rgb}{0,0.6,0}
\definecolor{codegray}{rgb}{0.5,0.5,0.5}
\definecolor{codepurple}{rgb}{0.58,0,0.82}

\usepackage{listings}
\lstdefinestyle{mystyle}{
    commentstyle=\color{codegreen},
    keywordstyle=\color{magenta},
    numberstyle=\tiny\color{codegray},
    stringstyle=\color{codepurple},
    basicstyle=\ttfamily,
    xleftmargin=.17\textwidth, xrightmargin=.15\textwidth,
    breakatwhitespace=false,
    breaklines=true,
    captionpos=b,
    keepspaces=true,
    numbersep=5pt,
    showspaces=false,
    showstringspaces=false,
    showtabs=false,
    tabsize=2
}
\lstset{style=mystyle}

\theoremstyle{plain}
\theorembodyfont{\normalfont}
\theoremseparator{~--}
\newtheorem*{proof}{Proof}
\newtheorem*{exam}{Example}
\renewcommand\qedsymbol{$\square$}

\newtheorem*{defi}{Definition}
\newtheorem{exo}{Exercise}%[section]
\newtheorem{ans}{Answer}%[section]
\newtheorem*{lemma}{Lemma}%[section]

\newcommand{\toto}{\twoheadrightarrow}

\newcommand{\rbeta}{\to_\beta}
\newcommand{\rsbeta}{\to_\beta^*}

\newcommand{\nil}{[\text{Nil}]}
\newcommand{\nth}{\textit{nth}}
\newcommand{\some}{[\textit{Some}]}
\newcommand{\none}{[\textit{None}]}

\newcommand{\Mlambda}{{\underline \Lambda}}
\newcommand{\mlambda}{{\underline \lambda}}
\newcommand{\mbeta}{{\underline \beta}}
\date{}
\newcommand{\tombeta}{\to_\mbeta}
\newcommand{\tosmbeta}{\to_\mbeta^*}

\title{$\lambda$-calculus}
\author{Valeran MAYTIE}

\begin{document}
  \maketitle

  \section*{Halting problem}

  \begin{center}
    There is no $\lambda$-term $H$ such has $H[t] = T$ if T has a normal
    form and $H[T] = F$ if $T$ has no normal form.
  \end{center}

  Let $N$ the set of $\lambda$-term that have a normal form.

  $N$ is not empty (it contains all the variables) and $N$ is not equal to
  $\Lambda$ because it not contains the $\lambda$-term $\Omega$.

  So we have $\Lambda \backslash N$ non-empty and non-equal to $\Lambda$

  $N$ and $\Lambda \backslash N$ are closed by the $\beta$-reduction.

  If $n, n' \in N$ such that $n =_\beta n'$ then $n' \in N$. Because $n'$
  has the same normal form as $n$.

  If $n, n' \in \Lambda \backslash N$ such that $n =_\beta n'$ then $n' \in
  \Lambda \backslash N$. Because $n$ has no normal form so $n'$ has no
  normal form.

  By Scott's theorem, the set $N$ and $\Lambda \backslash N$ are not recursively
  separable. So the $\lambda$-term $H$ does not exist.

  \section*{List in pure $\lambda$ calculus}

  We define this useful lambda term :

  \begin{multicols}{2}
    \begin{itemize}
      \item $[I] = \lambda x.x$
      \item $[T] = \lambda x\;y.x$
      \item $[F] = \lambda x\;y.y$
    \end{itemize}
    \columnbreak
    \begin{itemize}
      \item $\langle t, u \rangle = \lambda x. x\;t\;u$
      \item $\pi_1 \langle t, u \rangle = \langle t, u \rangle [T]$
      \item $\pi_2 \langle t, u \rangle = \langle t, u \rangle [F]$
    \end{itemize}
  \end{multicols}

  We define our integers as follows :

  \begin{multicols}{2}
    \begin{itemize}
      \item $[0] = [I]$
      \item $[S] = \lambda n. \langle [F], n\rangle$
      \item $[\textit{isZ}] = \lambda n. \pi_1\;t$
      \item $[P] = \lambda n. \pi_2\;t$
    \end{itemize}
  \end{multicols}

  We take the following fixpoint :

  \begin{align*}
    A &= (\lambda xy.y(xxy)) \\
    \Theta &= A\;A
  \end{align*}

  Finally, we define our lists as follows :

  \begin{itemize}
    \item $\nil = \lambda n\;f.\; n$
    \item $[x :: l] = \lambda n\;f.\; f\; x\; l$
  \end{itemize}

  \begin{center}
    The list $0 :: 1 :: \nil$ is represented as follows:
  \end{center}
  $$\lambda n_0\; f_0.\; f_0\; [0]\; (\lambda n_1\;f_1.\; f_1\;[1]\;(\lambda
  n\;f.\;n))$$

  The function \nth $\;k$ $l$ which return the
  $k^{nth}$ element of the list $l$ in an option type.

  The options are defined as follows :

  \begin{itemize}
    \item $[\textit{None}] = \lambda n s.\;n$
    \item $[\textit{Some}(x)] = \lambda n s.\;s x$
  \end{itemize}

  Now we can define \textit{nth} function :

    \begin{lstlisting}[caption=Nth function on list, mathescape=true]
nth = $\Theta$ ($\lambda$ f k l. l [None]
                    ($\lambda$ x l'. [isZ] k [Some(x)])
                                     (f ([P] k) l'))
    \end{lstlisting}

  We want to proof this property $\forall k\;l\;x, \textit{nth}\; [k]\; [l]
  =_\beta \textit{nth}\;[k+1]\;[x :: l]$.

  \proof We just need to compute one step of $\nth\;[k+1]\;[x::l]$
  \begin{align*}
    \nth\;[k+1]\;[x::[l]] &\rsbeta [x::[l]]\; \none\; 
    (\lambda x\;l.\;[\textit{isZ}]\; [k+1]\; [\textit{Some}(x)]\;
    (\nth\;[k]\;l)) \\
    &\rsbeta [x::[l]]\; \none\; 
    (\lambda x\;l.\;[F]\; [\textit{Some}(x)]\;
    (\nth\;[k]\;l)) \\
    &\rsbeta [x::[l]]\; \none\; 
    (\lambda x\;l.\;(\nth\;[k]\;l)) \\
    &\rsbeta (\lambda x\;l.\;(\nth\;[k]\;l))\;x\;[l]\\
    &\rsbeta \nth\;[k]\;[l]
  \end{align*}

  So by calculation, we have $\forall k\;l\;x, \textit{nth}\; [k]\; [l]
  =_\beta \textit{nth}\;[k+1]\;[x :: l]$.

  % We proceed by induction on $[k]$ :
  %
  % \begin{itemize}
  %   \item if $ k = 0$
  %     \begin{itemize}
  %       \item if $l = \nil$, then 
  %         \begin{multicols}{2}
  %         \begin{align*}
  %           \nth\; [0]\; \nil &\rsbeta \nil\; \none\; (\ldots) \\
  %                             &\rsbeta \none
  %         \end{align*}
  %
  %         \begin{align*}
  %           \nth\; [1]\; [x::\nil] &\rsbeta \nth\;[0]\;\nil \\
  %                                  &\rsbeta \none
  %         \end{align*}
  %         \end{multicols}
  %
  %       \item if $l = [x' :: [l]]$, then
  %         \begin{multicols}{2}
  %         \begin{align*}
  %           \nth\; [0]\; [x'::[l]] &\rsbeta
  %                             [\textit{isZ}]\; [0]\; [\textit{Some(x')}]\;
  %                             (\ldots) \\
  %                             &\rsbeta [\textit{Some(x')}]
  %         \end{align*}
  %
  %         \begin{align*}
  %           \nth\; [1]\; [x::x'::[l]] &\rsbeta
  %                             \nth\; [0]\; [x' :: [l]] \\
  %                             &\rsbeta [\textit{Some(x')}]
  %         \end{align*}
  %         \end{multicols}
  %     \end{itemize}
  %   \item if $k = n + 1$ by induction hypothesis we know that 
  %     $\forall l\;x, \nth\; [n]\; [l] =_\beta \nth\;[n+1]\;[x :: l]$.
  %
  %     We want to prove $\forall l\;x, \nth\; [k]\; [l] =_\beta \nth\;[k+1]\;[x :: l]$
  %
  %     \begin{itemize}
  %       \item if $l = \nil$:
  %
  %         \begin{multicols}{2}
  %           \begin{align*}
  %             \nth\; [k]\; \nil &\rsbeta \nil\; \none\; (\ldots) \\
  %                               &\rsbeta \none
  %           \end{align*}
  %
  %           \begin{align*}
  %             \nth\; [k + 1]\; [x::\nil] &\rsbeta \nth\;[k]\;\nil \\
  %                                        &\rsbeta \none
  %           \end{align*}
  %         \end{multicols}
  %
  %       \item if $l = x' :: l$
  %         \begin{align*}
  %           \nth\; [k + 1]\;[x::x':: l] &\rsbeta \nth\;[n+1]\;[x'::l] & &\text{One
  %           step on \nth} \\
  %           &=_\beta \nth\;[n]\;[l] & &\text{Induction Hypothesis}
  %         \end{align*}
  %     \end{itemize}
  % \end{itemize}
  \qedsymbol

\end{document}
