\documentclass{article}

\usepackage[utf8]{inputenc}
\usepackage{enumitem}
\usepackage{multirow}
\usepackage{xcolor}
\usepackage[T1]{fontenc}
% \usepackage[french]{babel}
\usepackage{hyperref}
\usepackage{amssymb}
\usepackage{mathtools}
\usepackage{ntheorem}
\usepackage{amsmath}
\usepackage{amssymb}
\usepackage[ a4paper, hmargin={2cm, 2cm}, vmargin={3cm, 3cm}]{geometry}
\usepackage{capt-of}
\usepackage{multicol}
\usepackage{mathpartir}

\usepackage[braket, qm]{qcircuit}
\usepackage{graphicx}

\usepackage{tikz}
\usetikzlibrary{angles,quotes, 3d}

\usepackage{hyperref}
\hypersetup{
    colorlinks,
    citecolor=black,
    filecolor=black,
    linkcolor=blue,
    urlcolor=blue
}

\usepackage{xcolor}

\definecolor{codegreen}{rgb}{0,0.6,0}
\definecolor{codegray}{rgb}{0.5,0.5,0.5}
\definecolor{codepurple}{rgb}{0.58,0,0.82}

\usepackage{listings}
\lstdefinestyle{mystyle}{
    commentstyle=\color{codegreen},
    keywordstyle=\color{magenta},
    numberstyle=\tiny\color{codegray},
    stringstyle=\color{codepurple},
    basicstyle=\ttfamily\footnotesize,
    breakatwhitespace=false,
    breaklines=true,
    captionpos=b,
    keepspaces=true,
    numbers=left,
    numbersep=5pt,
    showspaces=false,
    showstringspaces=false,
    showtabs=false,
    tabsize=2
}
\lstset{style=mystyle}

\theoremstyle{plain}
\theorembodyfont{\normalfont}
\theoremseparator{~--}
\newtheorem*{proof}{Proof}
\newtheorem*{exam}{Example}
\renewcommand\qedsymbol{$\square$}

\newtheorem*{defi}{Definition}
\newtheorem{exo}{Exercise}%[section]
\newtheorem{ans}{Answer}%[section]
\newtheorem*{lemma}{Lemma}%[section]

\newcommand{\toto}{\twoheadrightarrow}

\newcommand{\rbeta}{\to_\beta}
\newcommand{\rsbeta}{\to_\beta^*}

\newcommand{\Mlambda}{{\underline \Lambda}}
\newcommand{\mlambda}{{\underline \lambda}}
\newcommand{\mbeta}{{\underline \beta}}
\date{}
\newcommand{\tombeta}{\to_\mbeta}
\newcommand{\tosmbeta}{\to_\mbeta^*}

\title{$\lambda$-calculus}
\author{Valeran MAYTIE}

\begin{document}
  \maketitle

  \section*{Halting problem}

  \begin{center}
    There is no $\lambda$-term $H$ such has $H[t] = T$ if T has a normal
    form and $H[T] = F$ if $T$ has no normal form.
  \end{center}

  Let $N$ the set of $\lambda$-term that have a normal form.

  $N$ is not empty (it contains all the variables) and $N$ is not equal to
  $\Lambda$ because it not contains the $\lambda$-term $\Omega$.

  So we have $\Lambda \backslash N$ non-empty and non-equal to $\Lambda$

  By Scott's theorem, the set $N$ is not recursively separable. So the
  $\lambda$-term $H$ does not exist.

  \section*{List in pure $\lambda$ calculus}

  We define this useful lambda term :

  \begin{multicols}{2}
    \begin{itemize}
      \item $[I] = \lambda x.x$
      \item $[T] = \lambda x\;y.x$
      \item $[F] = \lambda x\;y.y$
    \end{itemize}
    \columnbreak
    \begin{itemize}
      \item $\langle t, u \rangle = \lambda x. x\;t\;u$
      \item $\pi_1 \langle t, u \rangle = \langle t, u \rangle [T]$
      \item $\pi_2 \langle t, u \rangle = \langle t, u \rangle [F]$
    \end{itemize}
  \end{multicols}

  We define our integers as follows :

  \begin{multicols}{2}
    \begin{itemize}
      \item $[0] = [I]$
      \item $[S] = \lambda n. \langle [F], n\rangle$
      \item $[\textit{isZ}] = \lambda n. \pi_1\;t$
      \item $[P] = \lambda n. \pi_2\;t$
    \end{itemize}
  \end{multicols}

  Finally, we define our lists as follows :

  \begin{itemize}
    \item $[] = \lambda n\;f.\; n$
    \item $x :: l = \lambda n\;f.\; f\; x\; l$
  \end{itemize}

  \begin{center}
    The list $0 :: 1 :: []$ is represented as follows:
  \end{center}
  $$\lambda n_0\; f_0.\; f_0\; [0]\; (\lambda n_1\;f_1.\; f_1\;[1]\;(\lambda
  n\;f.\;n))$$

\end{document}
