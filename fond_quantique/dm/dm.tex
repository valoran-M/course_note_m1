\documentclass[12pt]{article}
\usepackage[utf8]{inputenc}
\usepackage{textcomp,amssymb,amsmath,amsthm,stmaryrd}
\usepackage{bold-extra}
\usepackage[english] {babel}
\usepackage{color}
\usepackage{xcolor}
\usepackage{pdfpages}
\usepackage{environ}
\usepackage{verbatim}
\usepackage{enumerate}
\usepackage{multicol}
\usepackage{hyperref}
\hypersetup{colorlinks=true,linkcolor=blue}
\usepackage{fancyhdr}
\usepackage{mathpartir}

\usepackage{tikz}
\usepackage{quantikz}


\usepackage[a4paper, headheight=10mm, hmargin={2cm, 2cm}, vmargin={2cm, 2cm}]{geometry}

\newcommand{\x}{\noindent{\color{blue}xxx}\\}
\newcommand{\xx}[1]{\noindent{\color{blue}#1}\\}
\newcommand{\xxx}[2]{{\color{red}  $<$#2$>$ #1 $<\backslash $#2$>$ }}
\newcommand{\couic}[1]{}
\newcommand{\couicfootnote}[1]{}
\newcommand{\couicefootnote}[1]{}
\newcommand{\couicfootnotemark}{}

%\newcommand{i.e.}{{\em i.e. }}
%\newcommand{e.g.}{{\em e.g. }}
%\newcommand{etc.}{{\em etc. }}
%\newcommand{c.f.}{{\em cf. }}
\newcommand{\eg}{e.g.}
\newcommand{\etc}{etc.}
\newcommand{\ie}{i.e.}

\newcommand{\lra}{\longrightarrow}
\newcommand{\ul}{\underline}
\newcommand{\sub}[1]{\langle #1 \rangle}
\newcommand{\ve}[1]{\mathrm{\textbf{#1}}}

% \newcommand{\ket}[1]{| #1 \rangle}
% \newcommand{\bra}[1]{\langle #1 |}
% \newcommand{\braket}[2]{\langle #1 | #2 \rangle}
\newcommand{\trace}{\textrm{Tr}}

\newcommand{\dom}{\textrm{dom}}
\newcommand{\range}{\textrm{range}}

\newcommand{\Herm}{\textrm{Herm}^+_{d^2}(\mathbb{C})}
\newcommand{\herm}{\textrm{Herm}^+_{d}(\mathbb{C})}
\newcommand{\hermdm}{\textrm{Herm}^+_{dm}(\mathbb{C})}
\newcommand{\hermm}{\textrm{Herm}^+_{m}(\mathbb{C})}

\newcommand{\im}{\mathrm Im}
\newcommand{\ii}{\mathrm i}
\newcommand{\europ}{\textup{\emph{\geneuro}}}
\newcommand{\ad}{\operatorname{ad}}
\newcommand{\Id}{\operatorname{Id}}
\newcommand{\C}{\mathcal{C}}
\newcommand{\Z}{\mathbb{Z}}
\newcommand{\R}{\mathbb{R}}
\newcommand{\Supp}{{\bf\mathcal{S}}}
\newcommand{\pa}[1]{\left(#1\right)}
\newcommand{\acco}[1]{\left\{#1\right\}}
\newcommand{\norme}[1]{\left|\left|#1\right|\right|}

\newcommand{\tzero}{{\overline{0}}}
\newcommand{\size}{\textrm{\texttt{size}}}
\newcommand{\corch}[1]{[#1]}
\newcommand{\val}[1]{[\![#1]\!]}
\newcommand{\llaves}[1]{\{#1\}}
\newcommand{\true}{\mathrm{true}}
\newcommand{\false}{\mathrm{false}}
\newcommand{\type}{\colon\!}
\newcommand{\T}{\mathcal{T}}
\newcommand{\U}{\mathcal{U}}
\newcommand{\Sc}{\mathcal{S}}
\newcommand{\B}{\mathcal{B}}
\newcommand{\LL}{\mathcal{LL}}
\newcommand{\BLL}{\mathcal{BLL}}
\newcommand{\SL}{\ensuremath{\mathcal{SL}}}
\newcommand{\mapLLSL}[1]{\kappa_U({#1})}
\newcommand{\lasf}{\lambda 2^{la}}
\newcommand{\rulesf}{^\triangleleft}
\newcommand{\mapsf}{^\natural}
\newcommand{\Tsf}{\mathbb{T}(\lasf)}


\newcommand{\CNot}{\textit{CNot}}
\newcommand{\CSwap}{\textit{CSwap}}

\setlength{\topmargin} {-2.5cm}
\setlength{\oddsidemargin} {-1.1cm}
\setlength{\textwidth} {19cm}
\setlength{\textheight} {25cm}
\setlength{\parindent}{0pt}
\newcommand{\compresslist}{
\setlength{\topsep}{0pt}
\setlength{\itemsep}{0pt}
\setlength{\itemindent}{2em}
}

%Pour version avec corrections, décommenter les 3 lignes ci-dessous et commenter le %

%\NewEnviron
%\newenvironment{correct}
%{\color{red}}
%{}

% Pour version sans corrections à donner aux étudiants, commenter les 3 lignes ci-dessus et décommenter celle ci-dessous
\NewEnviron{correct}{}

\newtheoremstyle{exostyle}
  {0.2cm}{0.5cm}%                                 margin top and bottom
  {\rmfamily}%                                  text layout
  {0cm}%                                        indention of header
  {\bfseries}{ }%                               header font and text after
  {0cm}%                                        space after header
  {\thmname{#1}\thmnumber{ #2}:\thmnote{ #3}}%  header

\theoremstyle{exostyle}
 \newtheorem{exo}{Exercice}[section]
\renewcommand{\marks}[1]{{\small \color{magenta} \noindent \textbf{(#1 point(s) $\Box$)}
}}

\newcommand{\sem}[1]{\left\llbracket#1\right\rrbracket}
\newcommand{\cmp}[1]{\overline{#1}}

\lhead{Valeran MAYTIE}
\rhead{M1 MPRI T2}

\pagestyle{fancy}

\begin{document}

\begin{center}
{\bf {\Large  Homework}}\\
{\small \color{magenta} \textbf{(Guess : 40/50 points)}}
\end{center}
\section{Basic operation and their notation}

\exo[Inner/outer products in Dirac notation]

\begin{align*}
\left(\begin{array}{cccc}
 1 &0 \\
\end{array}\right)
\left(\begin{array}{c}
 1\\
 0 \\
\end{array}\right) =
\left(\begin{array}{c}
 1
\end{array}\right)
\quad
\left(\begin{array}{cccc}
 1 &0 \\
\end{array}\right)
\left(\begin{array}{c}
 0\\
 1
\end{array}\right) =
\left(\begin{array}{c}
 0\\
\end{array}\right)
\quad
\left(\begin{array}{cccc}
 1 &2 \\
\end{array}\right)
\left(\begin{array}{c}
 3\\
 4
\end{array}\right) =
\left(\begin{array}{c}
 11
\end{array}\right)
\end{align*}
The last one in Dirac notation :
\begin{align*}
  & (\bra{0} + 2\bra{1}) \times (3\ket 0 + 4 \ket 1) \\
  =& 3\braket 0 0 + 4 \braket 0 1 + 6 \braket 1 0 + 8 \braket 1 1 \\
  =& 3 + 8 \\
  =& 11
\end{align*}
\begin{align*}
\left(\begin{array}{c}
 1\\
 0
\end{array}\right)
\left(\begin{array}{cccc}
 1 &0 \\
\end{array}\right)
=
\left(\begin{array}{cccc}
 1 &0 \\
 0 &0 \\
\end{array}\right)
\qquad
\left(\begin{array}{c}
 0\\
 1
\end{array}\right)
\left(\begin{array}{cccc}
 1 &0 \\
\end{array}\right)
=
\left(\begin{array}{cccc}
 0 &0 \\
 1 &0 \\
\end{array}\right)
\qquad
\left(\begin{array}{c}
 3\\
 4
\end{array}\right)
\left(\begin{array}{cccc}
 1 &2 \\
\end{array}\right)
=
\left(\begin{array}{cccc}
 3 &6 \\
 4 &8 \\
\end{array}\right)
\end{align*}
The last two in Dirac notation :
\begin{align*}
  & (3\ket{0} + 4\ket{1}) \times (\bra 0 + 2\bra 1) \\
  =& 3 \ket 0 \bra 0 + 6 \ket 0 \bra 1 + 4 \ket 1 \bra 0 + 8 \ket 1 \bra 1
\end{align*}

\exo[Matrix products in Dirac notation]
\label{ex1.2:matrix}
\begin{align*}
\left(\begin{array}{cccc}
 0 &0 \\
 1 &0
\end{array}\right)
\left(\begin{array}{c}
 1\\
 0
\end{array}\right)
=
\left(\begin{array}{c}
 0\\
 1
\end{array}\right)
\qquad
\left(\begin{array}{cccc}
 0 &0\\
 1 &0
\end{array}\right)
\left(\begin{array}{c}
 0\\
 1
\end{array}\right)
=
\left(\begin{array}{c}
 0\\
 0
\end{array}\right)
\qquad
\left(\begin{array}{cccc}
 1 &3 \\
 2 &4
\end{array}\right)
\left(\begin{array}{c}
 5\\
 6
\end{array}\right)
=
\left(\begin{array}{c}
 23\\
 34
\end{array}\right)
\end{align*}
The last one in Dirac notation :
\begin{align*}
  &(\ket 0 \bra 0 + 3\ket 0 \bra 1 + 2 \ket 1 \bra 0 + 4 \ket 1 \bra 1)\times
  (5\ket 0 + 6 \ket 1) \\
  =&5\ket 0 \braket 0 0 + 6 \ket 0 \braket 0 1 + 15 \ket 0 \braket 1 0 + 18
    \ket 0 \braket 1 1 + \\
  & 10 \ket 1 \braket 0 0 + 12 \ket 1 \braket 0 1 + 20 \ket 1 \braket 1 0 +
    24 \ket 1 \braket 1 1\\
  =&5 \braket 0 0 \ket 0 + 18 \braket 1 1 \ket 0 + 10 \braket 0 0 \ket 1 +
  24 \braket 1 1 \ket 1 \\
  =&5 \ket 0 + 18 \ket 0 + 10 \ket 1 + 24 \ket 1 \\
  =&23  \ket 0 + 24 \ket 1
\end{align*}

\begin{mathpar}
\left(\begin{array}{cc}
 1& 0\\
 0& 1
\end{array}\right)
\left(\begin{array}{cc}
 1& 3\\
 2& 4
\end{array}\right)
=
\left(\begin{array}{cc}
 1& 3\\
 2& 4
\end{array}\right)
\and
\left(\begin{array}{cc}
 0& 0\\
 0& 1
\end{array}\right)
\left(\begin{array}{cc}
 0& 0\\
 0& 1
\end{array}\right)
=
\left(\begin{array}{cc}
 0& 0\\
 0& 1
\end{array}\right)
\newline
\left(\begin{array}{cc}
 1/\sqrt{2}& 1/\sqrt{2}\\
 1/\sqrt{2}& -1/\sqrt{2}
\end{array}\right)
\left(\begin{array}{cc}
 1/\sqrt{2}& 1/\sqrt{2}\\
 1/\sqrt{2}& -1/\sqrt{2}
\end{array}\right)
=
\left(\begin{array}{cc}
  1& 0\\
  0& 1
\end{array}\right)
\end{mathpar}
The last one in Dirac notation :
\begin{align*}
  &(1/\sqrt 2\ket 0 \bra 0 + 1/\sqrt 2\ket 0 \bra 1 + 1/\sqrt 2 \ket 1 \bra 0
  -1/\sqrt 2 \ket 1 \bra 1)^2 \\
  =& 1/2 \ket 0 \braket 0 0 \bra 0 + 1/2 \ket 0 \braket 0 0 \bra 1 + 1 /2 \ket 0
  \braket 1 1 \bra 0 - 1 / 2 \ket 0 \braket 1 1 \bra 1 \\
  & 1/2 \ket 1 \braket 0 0 \bra 1 + 1/2 \ket 1 \braket 0 0 \bra 0 - 1 /2 \ket 1
  \braket 1 1 \bra 0 + 1 /2 \ket 1 \braket 1 1 \bra 1\\
  =& 1/2 \ket 0 \bra 0 + 1 /2 \ket 0 \bra 0 +
  1/2 \ket 1 \bra 1 + 1 /2 \ket 1 \bra 1\\
  =& \ket 0 \bra 0 + \ket 1 \bra 1
\end{align*}

\newpage
\exo[Tensor products in Dirac/Coecke notation]
\begin{mathpar}
\left(\begin{array}{c}
 1\\
 0
\end{array}\right)
\otimes
\left(\begin{array}{c}
 0\\
 1
\end{array}\right)
=
\left(\begin{array}{c}
 1\\
 0\\
 0\\
 0
\end{array}\right)
\and
\left(\begin{array}{c}
 0\\
 1
\end{array}\right)
\otimes
\left(\begin{array}{c}
 1\\
 0
\end{array}\right)
=
\left(\begin{array}{c}
 0\\
 0\\
 1\\
 0
\end{array}\right)
\and
\left(\begin{array}{c}
 1\\
 2
\end{array}\right)
\otimes
\left(\begin{array}{c}
 3\\
 4
\end{array}\right)
=
\left(\begin{array}{c}
 3\\
 4\\
 6\\
 8
\end{array}\right)\\
\left(\begin{array}{c}
 1\\
 0
\end{array}\right)
\otimes
\left(\begin{array}{c}
 1\\
 0
\end{array}\right)
+
\left(\begin{array}{c}
 0\\
 1
\end{array}\right)
\otimes
\left(\begin{array}{c}
 0\\
 1
\end{array}\right)
=
\left(\begin{array}{c}
 1\\
 0\\
 0\\
 0
\end{array}\right)
+
\left(\begin{array}{c}
 0\\
 0\\
 0\\
 1
\end{array}\right)
=
\left(\begin{array}{c}
 1\\
 0\\
 0\\
 1
\end{array}\right)
\end{mathpar}
The last two in Dirac notation :
\begin{align*}
  (\ket 0 + 2 \ket 1) \otimes (3 \ket 0 + 4 \ket 1)
  =& 3 \ket{00} + 4 \ket{01} + 6 \ket{10} + 8 \ket{11}
\end{align*}
\begin{align*}
  \ket 0 \otimes \ket 0 + \ket 1 \otimes \ket 1
  =& \ket{00} + \ket{11}
\end{align*}

\begin{mathpar}
\left(\begin{array}{cc}
 1&0\\
 0&1
\end{array}\right)
\otimes
\left(\begin{array}{cc}
 1&0\\
 0&1
\end{array}\right)
=
\left(\begin{array}{cccc}
  1&0&0&0\\
  0&1&0&0\\
  0&0&1&0\\
  0&0&0&1
\end{array}\right)
\and
\left(\begin{array}{cc}
 1&0\\
 0&0
\end{array}\right)
\otimes
\left(\begin{array}{cc}
 1&0\\
 0&1
\end{array}\right)
=
\left(\begin{array}{cccc}
  1&0&0&0\\
  0&1&0&0\\
  0&0&0&0\\
  0&0&0&0
\end{array}\right)
\\
\left(\begin{array}{cc}
 1/\sqrt{2}& 1/\sqrt{2}\\
 1/\sqrt{2}& -1/\sqrt{2}
\end{array}\right)
\otimes
\left(\begin{array}{cc}
 1/\sqrt{2}& 1/\sqrt{2}\\
 1/\sqrt{2}& -1/\sqrt{2}
\end{array}\right)
=
\left(\begin{array}{cccc}
  1/2& 1/2& 1/2& 1/2\\
  1/2&-1/2& 1/2&-1/2\\
  1/2& 1/2&-1/2&-1/2\\
  1/2&-1/2&-1/2& 1/2
\end{array}\right)
\end{mathpar}

In Dirac notation :

\begin{align*}
  &(\ket 0 \bra 0 + \ket 1 \bra 1) \otimes (\ket 0 \bra 0 + \ket 1 \bra 1) \\
  =& \ket 0 \bra 0 \otimes \ket 0 \bra 0 + \ket 0 \bra 0 \otimes \ket 1 \bra 1 +
  \ket 1 \bra 1 \otimes \ket 0 \bra 0 + \ket 1 \bra 1 \otimes \ket 1 \bra 1 \\
  =& \ket 0 \bra 0 + \ket 1 \bra 1 + \ket 2 \bra 2 +\ket 3 \bra 3
\end{align*}
\begin{align*}
  &(\ket 0 \bra 0) \otimes (\ket 0 \bra 0 + \ket 1 \bra 1) \\
  =& \ket 0 \bra 0 \otimes \ket 0 \bra 0 + \ket 0 \bra 0 \otimes \ket 1 \bra 1
  \\
  =& \ket 0 \bra 0 + \ket 1 \bra 1
\end{align*}
\begin{align*}
  1/\sqrt 2  (&\ket 0 \bra 0 + \ket 0 \bra 1 + \ket 1 \bra 0 - \ket 1 \bra 1 )
    \otimes 1/\sqrt 2 (\ket 0 \bra 0 + \ket 1 \bra 0 + \ket 1 \bra 0 - \ket 1
    \bra 1) \\
  =1/2 (&\ket 0 \bra 0 + \ket 0 \bra 1 + \ket 0 \bra 2 + \ket 0 \bra 3 + \\
   &\ket 1 \bra 0 - \ket 1 \bra 1 + \ket 1 \bra 2 - \ket 1 \bra 3 + \\
   &\ket 2 \bra 0 + \ket 2 \bra 1 - \ket 2 \bra 2 - \ket 2 \bra 3 + \\
   &\ket 3 \bra 0 - \ket 3 \bra 1 - \ket 3 \bra 2 + \ket 3 \bra 3)
\end{align*}

\newpage

We want to prove $(A \otimes B) (C \otimes D) = (AC) \otimes (BD)$

\begin{itemize}
  \item Dirac's notation :

  \item Coecke's notation :

\end{itemize}

\exo[Dagger in Dirac/Coecke notation]

\begin{align*}
\left(\begin{array}{cc}
 1/\sqrt{2}& 1/\sqrt{2}\\
 1/\sqrt{2}& -1/\sqrt{2}
\end{array}\right)^\dagger
=
\left(\begin{array}{cc}
 1/\sqrt{2}& 1/\sqrt{2}\\
 1/\sqrt{2}& -1/\sqrt{2}
\end{array}\right)
\qquad
\left(\begin{array}{cc}
 1& 3i\\
 2& 4i
\end{array}\right)^\dagger
=
\left(\begin{array}{cc}
 1& 2\\
 -3i& -4i
\end{array}\right)
\end{align*}
In Dirac notation:
\begin{align*}
  &1/\sqrt 2 (\ket 0 \bra 0 + \ket 0 \bra 1 + \ket 1 \bra 0 - \ket 1 \bra
    1)^\dagger \\
  =&1/\sqrt 2 (\ket 0 \bra 0 + \ket 1 \bra 0 + \ket 0 \bra 1 - \ket 1 \bra 1)
\end{align*}
\begin{align*}
  &(\ket 0 \bra 0 + 3i \ket 0 \bra 1 + 2 \ket 1 \bra 0 + 4i \ket 1 \bra
    1)^\dagger \\
  =&(\ket 0 \bra 0 - 3i \ket 1 \bra 0 + 2 \ket 0 \bra 1 - 4i \ket 1 \bra
    1)
\end{align*}

\exo[Gates in Dirac notations]

$$ H = 1/\sqrt 2 (\ket 0 \bra 0+\ket 0 \bra 1+\ket 1 \bra 0-\ket 1 \bra 1)$$
$$ \textit{CNot} = \ket 0 \bra 0+\ket 1 \bra 1+\ket 3 \bra 2+\ket 2 \bra 3$$
$$ T = \ket 0 \bra 0 + e^{\frac{i\pi}{4}} \ket 1 \bra 1$$

Proof that are unitary matrix :

\begin{itemize}
  \item $H$ : $H^\dagger H = Id_1$ already do in Exercie-\ref{ex1.2:matrix}
  \item \CNot :
    \begin{align*}
      \CNot^\dagger \CNot &= 
        (\ket 0 \bra 0+\ket 1 \bra 1+\ket 3 \bra 2+\ket 2 \bra 3)
        (\ket 0 \bra 0+\ket 1 \bra 1+\ket 2 \bra 3+\ket 3 \bra 2) \\
        &=(\ket 0 \bra 0 + \ket 1 \bra 1 + \ket 2 \bra 2 + \ket 3 \bra 3) \\
        &= Id_4
    \end{align*}
  \item $T$:
    \begin{align*}
      (\ket 0 \bra 0 + e^{\frac{i\pi}{4}} \ket 1 \bra 1)^2
      &= \ket 0 \braket 0 0 \bra 0 + e^{\frac{i\pi}{4}} \ket 0 \braket 0 1 \bra 1
          + e^{\frac{i\pi}{4}}(\ket 1 \braket 1 0 \bra 0)
          + e^{\frac{i\pi}{2}}(\ket 1 \braket 1 1 \bra 1)\\
      &= \ket 0 \bra 0 + \ket 1 \bra 1
    \end{align*}
\end{itemize}

\exo[Pauli matrices in Dirac/Coecke notation]~

\begin{itemize}
  \item For all $i, k \in [0, 3]$ we want to show
    $\sigma_i \sigma_j = \delta_{ij} I + i \sum_k \epsilon_{ijk} \sigma_k$

    \begin{itemize}
      \item If $i=j$ then $\sigma_i \sigma_j = I$ and for all $k$ we have
        $\epsilon_{ijk} = 0$.

        So we have $\delta_{ij} I = I = \sigma_i \sigma_j$
    \end{itemize}

  \item For all $i, k \in [0, 3]$ we want to show $[\sigma_i, \sigma_j] = 
    2i \sum_k \epsilon_{ijk}$
    \begin{align*}
      [\sigma_i, \sigma_j] &= \sigma_i \sigma_j - \sigma_i \sigma_j \\
      &= (\delta_{ij} I + i \sum_k \epsilon_{ijk} \sigma_k) -
         (\delta_{ji} I + i \sum_k \epsilon_{jik} \sigma_k) \\
      &= i \sum_k \epsilon_{ijk} \sigma_k - i \sum_k \epsilon_{jik} \sigma_k \\
      &= i \sum_k \epsilon_{ijk} \sigma_k + i \sum_k \epsilon_{ijk} \sigma_k \\
      &= 2i \sum_k \epsilon_{ijk} \sigma_k
    \end{align*}
  \item For all $i, k \in [0, 3]$ with $i\not = j$ we want to show $\{\sigma_i, \sigma_j\} = 0$
    \begin{align*}
      \{\sigma_i, \sigma_j\} &= \sigma_i \sigma_j + \sigma_i \sigma_j \\
      &= \delta_{ij} I + i \sum_k \epsilon_{ijk} \sigma_k +
         \delta_{ji} I - i \sum_k \epsilon_{ijk} \sigma_k & \epsilon_{jik} =
         -\epsilon_{ijk}\\
      &= 2 \delta_{ij} I \\
      &= 0 & i \not = j
    \end{align*}
\end{itemize}



\documentclass{article}

\usepackage[utf8]{inputenc}
\usepackage{enumitem}
\usepackage{multirow}
\usepackage{xcolor}
\usepackage[T1]{fontenc}
% \usepackage[french]{babel}
\usepackage{hyperref}
\usepackage{amssymb}
\usepackage{mathtools}
\usepackage{ntheorem}
\usepackage{amsmath}
\usepackage{amssymb}
\usepackage[ a4paper, hmargin={2cm, 2cm}, vmargin={3cm, 3cm}]{geometry}
\usepackage{capt-of}
\usepackage{multicol}
\usepackage{mathpartir}

\usepackage[braket, qm]{qcircuit}
\usepackage{graphicx}

\usepackage{tikz}
\usetikzlibrary{angles,quotes, 3d}

\usepackage{hyperref}
\hypersetup{
    colorlinks,
    citecolor=black,
    filecolor=black,
    linkcolor=blue,
    urlcolor=blue
}

\usepackage{xcolor}

\definecolor{codegreen}{rgb}{0,0.6,0}
\definecolor{codegray}{rgb}{0.5,0.5,0.5}
\definecolor{codepurple}{rgb}{0.58,0,0.82}

\usepackage{listings}
\lstdefinestyle{mystyle}{
    commentstyle=\color{codegreen},
    keywordstyle=\color{magenta},
    numberstyle=\tiny\color{codegray},
    stringstyle=\color{codepurple},
    basicstyle=\ttfamily\footnotesize,
    breakatwhitespace=false,
    breaklines=true,
    captionpos=b,
    keepspaces=true,
    numbers=left,
    numbersep=5pt,
    showspaces=false,
    showstringspaces=false,
    showtabs=false,
    tabsize=2
}
\lstset{style=mystyle}

\theoremstyle{plain}
\theorembodyfont{\normalfont}
\theoremseparator{~--}
\newtheorem*{proof}{Proof}
\newtheorem*{exam}{Example}
\renewcommand\qedsymbol{$\square$}

\newtheorem*{defi}{Definition}
\newtheorem{exo}{Exercise}[section]
\newtheorem{ans}{Answer}[section]
\newtheorem*{lemma}{Lemma}%[section]

\newcommand{\toto}{\twoheadrightarrow}

\newcommand{\rbeta}{\to_\beta}
\newcommand{\rsbeta}{\to_\beta^*}

\newcommand{\Mlambda}{{\underline \Lambda}}
\newcommand{\mlambda}{{\underline \lambda}}
\newcommand{\mbeta}{{\underline \beta}}
\date{}
\newcommand{\tombeta}{\to_\mbeta}
\newcommand{\tosmbeta}{\to_\mbeta^*}

\title{$\lambda$-calculus}
\author{Valeran MAYTIE}

\begin{document}
  \maketitle

  \lemma \label{lemma:bar-phi}
  $\forall t, |t| \rsbeta \varphi(t)$

  \begin{center}
    \begin{tikzpicture}[line width=0.3mm, scale=1.4]
    \node(t) at (90:1) {$t$};
    \node(bt) at (210:1) {$|t|$};
    \node(pt) at (330:1) {$\varphi(t)$};

    \draw[->] (t) edge node[left] {$|\;|$} (bt)
              (t) edge node[right] {$\varphi$} (pt);
    \draw[->>, dotted] (bt) edge node[below] {$\beta$} (pt);
  \end{tikzpicture}
  \end{center}
  \textit{Proof}: We will show our property by induction on the term $t$.
  \begin{itemize}
    \item Case $t = x$, $|x| = x = \varphi(x)$, so we have $|t| \to_\beta^0
      \varphi(t)$.
    \item Case $t = \lambda x. t_0$, by the induction hypothesis we know that
      $|t_0| \rsbeta \varphi(t_0)$.
      \begin{align*}
        |(\lambda x. t_0)| &= \lambda x. |t_0| \\
        &\rsbeta \lambda x. \varphi(t_0) & \text{By induction hypothesis} \\
        &= \varphi(\lambda x. t_0)
      \end{align*}

    \item Case $t = \mlambda x. t_0$, by the induction hypothesis we know that
      $|t_0| \rsbeta \varphi(t_0)$.
      \begin{align*}
        |(\mlambda x. t_0)| &= \lambda x. |t_0| \\
        &\rsbeta \lambda x. \varphi(t_0) & \text{By induction hypothesis} \\
        &= \varphi(\mlambda x. t_0)
      \end{align*}

    \item Case $t = t_1\; t_2$ where $t_1 \not = \mlambda x. t_0$,
      by the induction hypothesis we know that
      $|t_1| \rsbeta \varphi(t_1)$ and $|t_2| \rsbeta \varphi(t_2)$.
      \begin{align*}
        |t_1\;t_2| &= |t_1|\;|t_2| \\
        &\rsbeta \varphi(t_1)\;\varphi(t_2) & \text{By induction hypothesis} \\
        &= \varphi(t_1\;t_2)
      \end{align*}

    \item Case $t=(\mlambda x. t_0)\;t_1$, by the induction hypothesis we know that
      $|t_0| \rsbeta \varphi(t_0)$ and $|t_1| \rsbeta \varphi(t_1)$.

        \begin{align*}
          |(\mlambda x. t_0)\;t_1| &= (\lambda x. |t_0|)\;|t_1| \\
            &\rsbeta (\lambda x. \varphi(t_0))\;\varphi(t_1) & \text{By
            induction hypothesis}\\
            &\to \varphi(t_0)\{x \leftarrow \varphi(t_1)\} \\
            &=\varphi((\mlambda x. t_0)\;t_1)
        \end{align*}

  \end{itemize}

  \qedsymbol
\end{document}

\section{Some mathematics}

\exo[Spectral theorems complements]~

Let $D$ and $D'$ two diagonal matrices and $U$ a unitary matrix.

\begin{itemize}
  \item Let $A = UDU^\dagger$ and $B=UD'U^\dagger$
    \begin{align*}
      AB &= UDU^\dagger UD'U^\dagger \\
        &= UDD'U^\dagger & \text{$U$ is a unitary matrix} \\
        &= UD'DU^\dagger & \text{$D$ and $D'$ are diagonal} \\
        &= UD'U^\dagger UDU^\dagger \\
        &= BA
    \end{align*}
  \item Let $M = UDU^\dagger$
    \begin{align*}
      MM^\dagger &= UDU^\dagger (UDU^\dagger)^\dagger \\
      &= UDU^\dagger (U^\dagger)^\dagger D^\dagger U^\dagger \\
      &= UDU^\dagger U D^\dagger U^\dagger \\
      &= UDD^\dagger U^\dagger \\
      &= UD^\dagger D U^\dagger \\
      &= UD^\dagger U^\dagger  U D U^\dagger \\
      &= (UD U^\dagger)^\dagger U D U^\dagger \\
      &= M^\dagger M
    \end{align*}
  \item Let $E = UDU^\dagger$ with having only non-negative value.

    Let $\ket{\psi} \in \mathcal M_{n,1}(\mathbb C)$,
      $d_i$ such that $E_{i,i} = d_i$
    \begin{align*}
      \bra{\psi} E \ket \psi &= \bra \psi U D U^\dagger \ket \psi \\
      &= (U^\dagger \ket \psi)^\dagger D (U^\dagger \ket \psi) \\
      &= \sum^n_{i=1} d_i (U_i \ket \psi)^2 \\
      &\geq 0
    \end{align*}

  \item Let $V = UDU^\dagger$ with $D$ having only modulus one values.
    \begin{align*}
      VV^\dagger &= UDU^\dagger(UDU^\dagger)^\dagger \\
      &= UDU^\dagger UD^\dagger U^\dagger \\
      &= UDD^\dagger U^\dagger \\
      &= UU^\dagger & \text{$D$ has only modulus one values}\\
      &= I
    \end{align*}
    So $V$ is a unitary matrix.

  \item Let $E$ a non-negative matrix. $E$ is spectrally decomposable with
    non-negative eigenvalues. We can take $M :=\sqrt E$ which is defined by its spectral decomposition being with the square roots
    of the eigenvalues of E. $M$ is hermitian and $E = MM$, so $E^\dagger =
    (MM)^\dagger = M^\dagger M^\dagger = MM = E$.

  \item The follow matrix is not normal :

    \[M=
      \left(\begin{tabular}{c c}
        1 & 2 \\
        3 & 4
      \end{tabular}\right)
    \]

    \[MM^\dagger = \left(\begin{tabular}{c c}
        5  & 11 \\
        11 & 25
      \end{tabular}\right) \not = 
      \left(\begin{tabular}{c c}
        10 & 14 \\
        14 & 20
      \end{tabular}\right) = M^\dagger M\]

\end{itemize}

\exo[Isometry versus unitary versus involution]~

\begin{itemize}
  \item Let $M$ a unitary and hermitian matrix.
    \begin{align*}
      M M &= M M^\dagger & M \text{is hermitian} \\
          &= I & M \text{ is unitary}
    \end{align*}
  \item Matrix $2\times 2$ unitary that is not an involution :
    \begin{align*}
      M = \left(
      \begin{array}{c c}
        1 & 0\\
        0 & i\\
      \end{array}
      \right)
    \end{align*}
    The inverse of $M$ is :
    \begin{align*}
      M^{-1} = \left(
      \begin{array}{c c}
        1 & 0\\
        0 & -i\\
      \end{array}
      \right)
    \end{align*}
    We have $M \not = M^{-1}$ so $M$ is not an involution.
  \item Matrix $m\times n$ isometry that is not a unitary:
    \begin{align*}
      M = \left(
      \begin{array}{c c}
        0 & 1\\
      \end{array}
      \right)
    \end{align*}
    \begin{align*}
      M^\dagger M =
      \left(\begin{array}{c}
        1\\
      \end{array}
      \right) = I_1
      \quad
      MM^\dagger =
      \left(\begin{array}{c c}
        0 & 0\\
        0 & 1
      \end{array}
      \right) \not = I_2
    \end{align*}
  \item Let $M$ an $n \times n$ isometry matrix ($M^\dagger M = I_n$)
    \begin{align*}
      MM^\dagger &=
      \sum_{i,j} M_{i,j} \ket i \bra j \sum_{i,j} M_{i,j} \ket j \bra i \\
      &= \sum_{i,j} M_{i,j} \ket{ij} \bra{ji} \\
      &= \sum_{i,j} M_{i,j} \ket{ji} \bra{ij} \\
      &= \sum_{i,j} M_{i,j} \ket{j} \bra{i} \sum_{i,j} M_{i,j} \ket i \bra j \\
      &= M^\dagger M \\
      &= I_n
    \end{align*}

\end{itemize}

\section{On the nature of quantum information}

\exo[Hadamard]~

\begin{itemize}
  \item $a = 0$
    \begin{align*}
      H \ket 0 &= \frac 1 {\sqrt 2}(\ket 0 + \ket 1)\\
      &= \frac 1 {\sqrt 2}(\ket 0 + (-1)^0 \ket 1)
    \end{align*}
  \item $a = 1$
    \begin{align*}
      H \ket 1 &= \frac 1 {\sqrt 2}(\ket 0 - \ket 1)\\
      &= \frac 1 {\sqrt 2}(\ket 0 + (-1)^1 \ket 1)
    \end{align*}
\end{itemize}

\exo[Who controls whom?]~

We define :
\begin{align*}
\textit{NotC} &=
\left(\begin{array}{c c c c}
  1 & 0 & 0 & 0\\
  0 & 0 & 0 & 1\\
  0 & 0 & 1 & 0\\
  0 & 1 & 0 & 0
\end{array}\right)
\end{align*}
\begin{align*}
\textit{NotC}\ket{00} &= \ket{00}&
\textit{NotC}\ket{01} &= \ket{11} \\
\textit{NotC}\ket{10} &= \ket{10}&
\textit{NotC}\ket{11} &= \ket{01} \\
\end{align*}

We want to proof $(H \otimes H) \CNot (H \otimes H) = \textit{NotC}$ :
\begin{align*}
  (H \otimes H) (\ket x \otimes \ket y) &= (H \ket x \otimes H \ket y) \\
  &= \frac 1 {\sqrt 2}(\ket 0 + (-1)^x \ket 1) \otimes
     \frac 1 {\sqrt 2}(\ket 0 + (-1)^y \ket 1) \\
  &= \frac 1 2 (\ket{00} + (-1)^x \ket{10} + (-1)^y\ket{01} + (-1)^{x+y}\ket{11})
\end{align*}
We apply the operator \CNot :
\begin{align*}
  &\CNot (\frac 1 2 (\ket{00} + (-1)^x \ket{10} + (-1)^y\ket{01} +
  (-1)^{x+y}\ket{11})) \\
  &=\frac 1 2 (\ket{00} + (-1)^x \ket{11} + (-1)^y\ket{01} + (-1)^{x+y}\ket{10})
  \\
  &= \frac 1{\sqrt 2}(\ket 0 + (-1)^{x+y}\ket 1)\otimes (\frac 1{\sqrt 2}(\ket 0 +
  (-1)^y\ket 1)) \\
  &= H\ket{x \oplus y} \otimes H\ket{y} & (-1)^{x+y} = (-1)^{x\oplus y}\; (x
  \oplus y \in \{0, 1\})
\end{align*}
Finlay we apply $(H \otimes H)$:
\begin{align*}
  (H \otimes H) (H\ket{x\oplus y} \otimes H\ket y)
  &= HH\ket{x\oplus y} \otimes HH\ket y \\
  &= \ket{x\oplus y} \otimes \ket y\\
  &= \textit{NotC}(\ket x \otimes \ket y)
\end{align*}

In quantum circuit :
\begin{center}
\begin{quantikz}
& \gate{H} & \ctrl{1} & \gate{H} & \qw \\
& \gate{H} & \targ{}  & \gate{H} & \qw
\end{quantikz}=\begin{quantikz}
  & \targ{}   & \qw \\
  & \ctrl{-1} & \qw
\end{quantikz}
\end{center}


\section{Protocols}

\exo[Canonical basis versus diagonal basis]~

\begin{itemize}
  \item If Bob measures the result in the same basis then he can retrieve the
    information sent

  \item If Bob measures in an other basis then he learns nothing about the
    message.

  \item If Eve intercepts and measures it in the same basis then Bob can have
    some information on the message if he read the message in the same basis.

  \item But if Eve intercepts and measures it in an other basis and Bob read the
    message in the original basis then he learns nothing about the message.
\end{itemize}

\exo[BB84]~

\begin{enumerate}
  \item Alice will start by producing a random string of bits, encode each of
    them either into the canonical or the diagonal basis, and send that to Bob.

  \item Bob will measure them either using the canonical basis or the
    diagonal basis, at random.

  \item Bob will broadcast which bases he used

  \item Alice will know when Bob used the same base. When Bob has used the right
    base, Bob's information is correct, otherwise it is wrong (previous exercise).

  \item Eve does not know the bases like Bob And she has very little chance of
    having the right basic sequence ($\frac{1}{2^n}$). But she's going to
    disrupt Bob's measurements.

  \item 
\end{enumerate}

They can use common measurements bases to create an encryption key. For example,
a basic measurements base sequence can become a binary code with 0 when we have
the base $\mathcal M$ and 1 if we have the base $\mathcal M'$. With this
generate key we can communicate with an existing encryption protocol.

\exo[Quantum random access code]~

TODO

\exo[The Bell basis]~

\begin{align*}
  \ket{\beta_0} &= \frac{1}{\sqrt 2}(\ket{00} + \ket{11})
  &\ket{\beta_1} &= (X \otimes I)\ket{\beta_0} =
    \frac{1}{\sqrt 2}(\ket{10}+\ket{01}) \\
  \ket{\beta_2} &= (Y \otimes I)\ket{\beta_2} =
    \frac{i}{\sqrt 2}(\ket{10} - \ket{01})
  &\ket{\beta_3} &= (Z \otimes I)\ket{\beta_0} =
    \frac{1}{\sqrt 2}(\ket{00}-\ket{11})
\end{align*}

These four states are orthogonal and orthonormal,

orthogonal:
\begin{align*}
  \braket{\beta_0}{\beta_1} &= \frac 1 2 \times 0 = 0 &
  \braket{\beta_0}{\beta_2} &= \frac i 2 \times 0 = 0 \\
  \braket{\beta_0}{\beta_3} &= \frac 1 2 - \frac 1 2 = 0 &
  \braket{\beta_1}{\beta_2} &= \frac i 2 - \frac i 2 = 0 \\
  \braket{\beta_1}{\beta_3} &= \frac 1 2 \times 0 = 0 &
  \braket{\beta_2}{\beta_3} &= \frac i 2 \times 0 = 0
\end{align*}

orthonormal:
\begin{align*}
  \braket{\beta_0}{\beta_0} &= \frac 1 2 + \frac 1 2 = 1 &
  \braket{\beta_1}{\beta_1} &= \frac 1 2 + \frac 1 2 = 1 \\
  \braket{\beta_2}{\beta_2} &= \frac 1 2 + \frac 1 2 = 1 &
  \braket{\beta_3}{\beta_3} &= \frac 1 2 + \frac 1 2 = 1 \\
\end{align*}

So, this states are an orthonormal basis.

It is also a valid measurement :

\begin{align*}
  \sum_i \mathcal M _i &=
  \ket {\beta_0} \bra {\beta_0} + \ket {\beta_1} \bra {\beta_1} +
  \ket {\beta_2} \bra {\beta_2} + \ket {\beta_3} \bra {\beta_3} \\
  &= I_4
\end{align*}

\exo[Superdense coding]~

At the beginning, Alice and Bob share an entangled state $\ket{\beta_0} = \frac
1 {\sqrt 2}(\ket{00} + \ket{11}$ Alice can change $\ket{\beta_0}$ in $\ket
{\beta_k}$ with this operation :

\begin{itemize}
  \item $\ket{\beta_0}$ : do nothing
  \item $\ket{\beta_1}$ : apply the matrix $X$
    \begin{align*}
      X\ket{\beta_0} &= (X \otimes I) \frac{1}{\sqrt 2}(\ket{00} + \ket{11}) \\
      &= \frac{1}{\sqrt 2}(\ket{10} + \ket{01}) \\
      &= \ket{\beta_1}
    \end{align*}
  \item $\ket{\beta_2}$ : apply the matrix $Y$
    \begin{align*}
      Y\ket{\beta_0} &= (Y \otimes I) \frac{1}{\sqrt 2}(\ket{00} + \ket{11}) \\
      &= \frac{i}{\sqrt 2}(\ket{10} - \ket{01}) \\
      &= \ket{\beta_2}
    \end{align*}
  \item $\ket{\beta_3}$ : apply the matrix $Z$
    \begin{align*}
      Y\ket{\beta_0} &= (Z \otimes I) \frac{1}{\sqrt 2}(\ket{00} + \ket{11}) \\
      &= \frac{1}{\sqrt 2}(\ket{00} - \ket{11}) \\
      &= \ket{\beta_3}
    \end{align*}
\end{itemize}

So Alice can encode the four possible pairs of bits (00, 01, 10 and 11) with the
4 Bell states. We have shown that Alice can modify $\ket{B_0}$ on her own, so
with  qubit she can change the communication state to one of the 4 states. Bob
can measure the result and retrieve the information from Alice.

\exo[Discussion: classical description of a single qubit]~

A qubit is coded with this formula : $\alpha \ket 0 + \beta \ket 1$.
We just need to send $2$ complexes numbers. So if a number is encoded with $n$
bits we send $4n$ bits.


\exo[Teleportation]~

\begin{align*}
  \frac 1 2 \sum_i \ket{\beta_i} \otimes \sigma_i \ket{\psi}
  =& \frac{1}{2\sqrt 2}((\ket{00}+\ket{11}) \otimes (\alpha \ket 0 + \beta \ket 1)+\\
  &(\ket{10}+\ket{01}) \otimes (\alpha \ket 1 + \beta \ket 0)+\\
  &(i\ket{10} - i \ket{01}) \otimes (i\alpha \ket 1 - i\beta \ket 0)+\\
  &(\ket{00} - \ket{11}) \otimes (\alpha \ket 0 - \beta \ket 1)) \\
  =& \frac{1}{\sqrt 2}(\alpha \ket{000} + \beta \ket{100} + \alpha \ket{011} +
  \beta \ket{111} \\
  =& \frac{1}{2\sqrt 2}(2\alpha(\ket{000} + \ket{011}) + 2\beta(\ket{100} +
  \ket{111})) \\
  =& \frac{1}{\sqrt 2}(\alpha(\ket{000} + \ket{011}) + \beta(\ket{100} + \ket{111}))
    \\
  =& \frac{1}{\sqrt 2}(
    (\alpha \ket{0} + \beta \ket{1}) \otimes \ket 0 \otimes \ket 0 + 
    (\alpha \ket{0} + \beta \ket{1}) \otimes \ket 1 \otimes \ket 1) \\
  =& \frac{1}{\sqrt 2}(\ket{\psi} \otimes \ket{0} \otimes \ket{0}) +
    (\ket{\psi} \otimes \ket 1 \otimes \ket 1)
\end{align*}

\exo[The swap test]~

Before the measurement we have this state :

\begin{align*}
  \ket{\kappa}&=(H \otimes I \otimes I) \CSwap (H \otimes I \otimes I) \ket{0} \otimes
  \ket{\phi} \otimes \ket{\psi} \\
  &= (H \otimes I \otimes I) \CSwap ((\frac{1}{\sqrt 2}(\ket{0} + \ket{1}))
  \otimes \ket{\phi} \otimes \ket{\psi}) \\
  &= (H \otimes I \otimes I) \CSwap \frac{1}{\sqrt 2}
     (\ket{0\phi\psi} + \ket{1\phi\psi}) \\
  &= (H \otimes I \otimes I)\frac{1}{\sqrt 2}
     (\ket{0\phi\psi} + \ket{1\psi\phi}) \\
  &= \frac 1 2(\ket 0 \otimes (\ket{\phi\psi} + \ket{\psi\phi}) -
               \ket 1 \otimes (\ket{\phi\psi} + \ket{\psi\phi})) \\
  &= \frac 1 2(\ket{0\phi\psi} + \ket{0\psi\phi} +
               \ket{1\phi\psi} - \ket{1\psi\phi})
\end{align*}

We apply the measures :

\begin{align*}
  p(0) &= \bra \kappa \ket 0 \bra 0 \ket \kappa \\
  &= \frac 1 2 (\bra{\phi\psi} + \ket{\psi\phi}) \times
     \frac 1 2 (\bra{\phi\psi} + \ket{\psi\phi}) \\
  &= \frac 1 4 (2 + \braket{\phi\psi}{\psi\phi} + \braket{\psi\phi}{\phi\psi})\\
  &= \frac 1 4 (2 + 2\braket{\phi\psi}{\psi\phi})\\
  &= \frac 1 2 + \frac{\braket{\psi\phi}{\psi\phi}}{2}\\
  &= \frac 1 2 + \frac 1 2 |\braket{\psi}{\phi}|^2\\
\end{align*}

we have $p(1) = 1 - p(0)$ so $p(1) = \frac 1 2 - \frac 1 2
|\braket{\psi}{\phi}|^2$

\exo[Quantum fingerprinting]~

TODO




\section{Quantum error correction}

\section{Bell}

\exo[Probability of winning the CHSH game]~

We have 4 case :

\begin{itemize}
  \item s = r = 0

    \begin{align*}
      \mathcal P(\text{win}) =& \mathcal P(a = b = 0) + \mathcal P(a = b = 1) \\
      =& |((\cos(0) \bra 0 + \sin(0) \bra 1) \otimes
          ((\cos(\frac \pi 8) \bra 0) + \sin(\frac \pi 8) \bra 1)) \ket{\beta_0}|^2
          + \\
      & |((\sin(0) \bra 0 - \cos(0) \bra 1) \otimes
          ((\sin(\frac \pi 8) \bra 0) - \cos(\frac \pi 8) \bra 1))\ket{\beta_0}|^2
          \\
      =& |(\cos(\frac \pi 8)\bra{00} + \sin(\frac \pi 8)\bra{01})
      \ket{\beta_0}|^2 +
         |(-\sin(\frac \pi 8)\bra{10} + \cos(\frac \pi 8)\bra{11}) \ket{\beta_0}|^2
      \\
      =& |\frac{1}{\sqrt 2} \cos(\frac \pi 8)|^2 +
         |\frac{1}{\sqrt 2} \cos(\frac \pi 8)|^2 \\
      =& \frac 1 2 \cos^2(\frac \pi 8) + \frac 1 2 \cos^2(\frac \pi 8) \\
      =& \cos^2(\frac \pi 8)
    \end{align*}

  \item the following calculations are similar and we obtain
    $\cos^2(\frac \pi 8)$
\end{itemize}

There are 4 different ways to draw $s$ and $r$ :

$$ \mathcal P(\text{win}) = 4 \times \frac 1 4 \times \cos^2(\frac \pi 8)
= \cos^2(\frac \pi 8)$$


\end{document}
