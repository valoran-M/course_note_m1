\documentclass[12pt]{article}
\usepackage{textcomp,amssymb,amsmath,amsthm,stmaryrd}
\usepackage{bold-extra}
\usepackage[utf8]{inputenc}
\usepackage[english] {babel}
\usepackage[pdftex]{graphicx,color}
\usepackage{environ}
\usepackage{verbatim}
\usepackage{enumerate}
\usepackage{multicol}
\usepackage{hyperref}
\hypersetup{colorlinks=true,linkcolor=blue}
\usepackage{fancyhdr}
\usepackage{mathpartir}

\usepackage{tikz}

\usepackage[a4paper, headheight=10mm, hmargin={2cm, 2cm}, vmargin={2cm, 2cm}]{geometry}

\newcommand{\x}{\noindent{\color{blue}xxx}\\}
\newcommand{\xx}[1]{\noindent{\color{blue}#1}\\}
\newcommand{\xxx}[2]{{\color{red}  $<$#2$>$ #1 $<\backslash $#2$>$ }}
\newcommand{\couic}[1]{}
\newcommand{\couicfootnote}[1]{}
\newcommand{\couicefootnote}[1]{}
\newcommand{\couicfootnotemark}{}

%\newcommand{i.e.}{{\em i.e. }}
%\newcommand{e.g.}{{\em e.g. }}
%\newcommand{etc.}{{\em etc. }}
%\newcommand{c.f.}{{\em cf. }}
\newcommand{\eg}{e.g.}
\newcommand{\etc}{etc.}
\newcommand{\ie}{i.e.}

\newcommand{\lra}{\longrightarrow}
\newcommand{\ul}{\underline}
\newcommand{\sub}[1]{\langle #1 \rangle}
\newcommand{\ve}[1]{\mathrm{\textbf{#1}}}

\newcommand{\ket}[1]{| #1 \rangle}
\newcommand{\bra}[1]{\langle #1 |}
\newcommand{\braket}[2]{\langle #1 | #2 \rangle}
\newcommand{\trace}{\textrm{Tr}}

\newcommand{\dom}{\textrm{dom}}
\newcommand{\range}{\textrm{range}}

\newcommand{\Herm}{\textrm{Herm}^+_{d^2}(\mathbb{C})}
\newcommand{\herm}{\textrm{Herm}^+_{d}(\mathbb{C})}
\newcommand{\hermdm}{\textrm{Herm}^+_{dm}(\mathbb{C})}
\newcommand{\hermm}{\textrm{Herm}^+_{m}(\mathbb{C})}

\newcommand{\im}{\mathrm Im}
\newcommand{\ii}{\mathrm i}
\newcommand{\europ}{\textup{\emph{\geneuro}}}
\newcommand{\ad}{\operatorname{ad}}
\newcommand{\Id}{\operatorname{Id}}
\newcommand{\C}{\mathcal{C}}
\newcommand{\Z}{\mathbb{Z}}
\newcommand{\R}{\mathbb{R}}
\newcommand{\Supp}{{\bf\mathcal{S}}}
\newcommand{\pa}[1]{\left(#1\right)}
\newcommand{\acco}[1]{\left\{#1\right\}}
\newcommand{\norme}[1]{\left|\left|#1\right|\right|}

\newcommand{\tzero}{{\overline{0}}}
\newcommand{\size}{\textrm{\texttt{size}}}
\newcommand{\corch}[1]{[#1]}
\newcommand{\val}[1]{[\![#1]\!]}
\newcommand{\llaves}[1]{\{#1\}}
\newcommand{\true}{\mathrm{true}}
\newcommand{\false}{\mathrm{false}}
\newcommand{\type}{\colon\!}
\newcommand{\T}{\mathcal{T}}
\newcommand{\U}{\mathcal{U}}
\newcommand{\Sc}{\mathcal{S}}
\newcommand{\B}{\mathcal{B}}
\newcommand{\LL}{\mathcal{LL}}
\newcommand{\BLL}{\mathcal{BLL}}
\newcommand{\SL}{\ensuremath{\mathcal{SL}}}
\newcommand{\mapLLSL}[1]{\kappa_U({#1})}
\newcommand{\lasf}{\lambda 2^{la}}
\newcommand{\rulesf}{^\triangleleft}
\newcommand{\mapsf}{^\natural}
\newcommand{\Tsf}{\mathbb{T}(\lasf)}


\newcommand{\CNot}{\textit{CNot}}
\newcommand{\CSwap}{\textit{CSwap}}

\setlength{\topmargin} {-2.5cm}
\setlength{\oddsidemargin} {-1.1cm}
\setlength{\textwidth} {19cm}
\setlength{\textheight} {25cm}
\setlength{\parindent}{0pt}
\newcommand{\compresslist}{
\setlength{\topsep}{0pt}
\setlength{\itemsep}{0pt}
\setlength{\itemindent}{2em}
}

%Pour version avec corrections, décommenter les 3 lignes ci-dessous et commenter le %

%\NewEnviron
%\newenvironment{correct}
%{\color{red}}
%{}

% Pour version sans corrections à donner aux étudiants, commenter les 3 lignes ci-dessus et décommenter celle ci-dessous
\NewEnviron{correct}{}

\newtheoremstyle{exostyle}
  {0.2cm}{0.5cm}%                                 margin top and bottom
  {\rmfamily}%                                  text layout
  {0cm}%                                        indention of header
  {\bfseries}{ }%                               header font and text after
  {0cm}%                                        space after header
  {\thmname{#1}\thmnumber{ #2}:\thmnote{ #3}}%  header

\theoremstyle{exostyle}
 \newtheorem{exo}{Exercice}[section]
\renewcommand{\marks}[1]{{\small \color{magenta} \noindent \textbf{(#1 point(s) $\Box$)}
}}

\newcommand{\sem}[1]{\left\llbracket#1\right\rrbracket}
\newcommand{\cmp}[1]{\overline{#1}}

\lhead{Valeran MAYTIE}
\rhead{M1 MPRI T2}

\pagestyle{fancy}

\begin{document}

\begin{center}
{\bf {\Large  Homework}}\\
\end{center}

\section{Basic operation and their notation}

\exo[Inner/outer products in Dirac notation]

\begin{align*}
\left(\begin{array}{cccc}
 1 &0 \\
\end{array}\right)
\left(\begin{array}{c}
 1\\
 0 \\
\end{array}\right) =
\left(\begin{array}{c}
 1
\end{array}\right)
\quad
\left(\begin{array}{cccc}
 1 &0 \\
\end{array}\right)
\left(\begin{array}{c}
 0\\
 1
\end{array}\right) =
\left(\begin{array}{c}
 0\\
\end{array}\right)
\quad
\left(\begin{array}{cccc}
 1 &2 \\
\end{array}\right)
\left(\begin{array}{c}
 3\\
 4
\end{array}\right) =
\left(\begin{array}{c}
 11
\end{array}\right)
\end{align*}
The last one in Dirac notation :
\begin{align*}
  & (\bra{0} + 2\bra{1}) \times (3\ket 0 + 4 \ket 1) \\
  =& 3\braket 0 0 + 4 \braket 0 1 + 6 \braket 1 0 + 8 \braket 1 1 \\
  =& 3 + 8 \\
  =& 11
\end{align*}
\begin{align*}
\left(\begin{array}{c}
 1\\
 0
\end{array}\right)
\left(\begin{array}{cccc}
 1 &0 \\
\end{array}\right)
=
\left(\begin{array}{cccc}
 1 &0 \\
 0 &0 \\
\end{array}\right)
\qquad
\left(\begin{array}{c}
 0\\
 1
\end{array}\right)
\left(\begin{array}{cccc}
 1 &0 \\
\end{array}\right)
=
\left(\begin{array}{cccc}
 0 &0 \\
 1 &0 \\
\end{array}\right)
\qquad
\left(\begin{array}{c}
 3\\
 4
\end{array}\right)
\left(\begin{array}{cccc}
 1 &2 \\
\end{array}\right)
=
\left(\begin{array}{cccc}
 3 &6 \\
 4 &8 \\
\end{array}\right)
\end{align*}
The last two in Dirac notation :
\begin{align*}
  & (3\ket{0} + 4\ket{1}) \times (\bra 0 + 2\bra 1) \\
  =& 3 \ket 0 \bra 0 + 6 \ket 0 \bra 1 + 4 \ket 1 \bra 0 + 8 \ket 1 \bra 1
\end{align*}

\exo[Matrix products in Dirac notation]
\begin{align*}
\left(\begin{array}{cccc}
 0 &0 \\
 1 &0
\end{array}\right)
\left(\begin{array}{c}
 1\\
 0
\end{array}\right)
=
\left(\begin{array}{c}
 0\\
 1
\end{array}\right)
\qquad
\left(\begin{array}{cccc}
 0 &0\\
 1 &0
\end{array}\right)
\left(\begin{array}{c}
 0\\
 1
\end{array}\right)
=
\left(\begin{array}{c}
 0\\
 0
\end{array}\right)
\qquad
\left(\begin{array}{cccc}
 1 &3 \\
 2 &4
\end{array}\right)
\left(\begin{array}{c}
 5\\
 6
\end{array}\right)
=
\left(\begin{array}{c}
 23\\
 34
\end{array}\right)
\end{align*}
The last one in Dirac notation :
\begin{align*}
  &(\ket 0 \bra 0 + 3\ket 0 \bra 1 + 2 \ket 1 \bra 0 + 4 \ket 1 \bra 1)\times
  (5\ket 0 + 6 \ket 1) \\
  =&5\ket 0 \braket 0 0 + 6 \ket 0 \braket 0 1 + 15 \ket 0 \braket 1 0 + 18
    \ket 0 \braket 1 1 + \\
  & 10 \ket 1 \braket 0 0 + 12 \ket 1 \braket 0 1 + 20 \ket 1 \braket 1 0 +
    24 \ket 1 \braket 1 1\\
  =&5 \braket 0 0 \ket 0 + 18 \braket 1 1 \ket 0 + 10 \braket 0 0 \ket 1 +
  24 \braket 1 1 \ket 1 \\
  =&5 \ket 0 + 18 \ket 0 + 10 \ket 1 + 24 \ket 1 \\
  =&23  \ket 0 + 24 \ket 1
\end{align*}

\begin{mathpar}
\left(\begin{array}{cc}
 1& 0\\
 0& 1
\end{array}\right)
\left(\begin{array}{cc}
 1& 3\\
 2& 4
\end{array}\right)
=
\left(\begin{array}{cc}
 1& 3\\
 2& 4
\end{array}\right)
\and
\left(\begin{array}{cc}
 0& 0\\
 0& 1
\end{array}\right)
\left(\begin{array}{cc}
 0& 0\\
 0& 1
\end{array}\right)
=
\left(\begin{array}{cc}
 0& 0\\
 0& 1
\end{array}\right)
\newline
\left(\begin{array}{cc}
 1/\sqrt{2}& 1/\sqrt{2}\\
 1/\sqrt{2}& -1/\sqrt{2}
\end{array}\right)
\left(\begin{array}{cc}
 1/\sqrt{2}& 1/\sqrt{2}\\
 1/\sqrt{2}& -1/\sqrt{2}
\end{array}\right)
=
\left(\begin{array}{cc}
  1& 0\\
  0& 1
\end{array}\right)
\end{mathpar}
The last one in Dirac notation :
\begin{align*}
  &(1/\sqrt 2\ket 0 \bra 0 + 1/\sqrt 2\ket 0 \bra 1 + 1/\sqrt 2 \ket 1 \bra 0
  -1/\sqrt 2 \ket 1 \bra 1)^2 \\
  =& 1/2 \ket 0 \braket 0 0 \bra 0 + 1/2 \ket 0 \braket 0 0 \bra 1 + 1 /2 \ket 0
  \braket 1 1 \bra 0 - 1 / 2 \ket 0 \braket 1 1 \bra 1 \\
  & 1/2 \ket 1 \braket 0 0 \bra 1 + 1/2 \ket 1 \braket 0 0 \bra 0 - 1 /2 \ket 1
  \braket 1 1 \bra 0 + 1 /2 \ket 1 \braket 1 1 \bra 1\\
  =& 1/2 \ket 0 \bra 0 + 1 /2 \ket 0 \bra 0 +
  1/2 \ket 1 \bra 1 + 1 /2 \ket 1 \bra 1\\
  =& \ket 0 \bra 0 + \ket 1 \bra 1
\end{align*}

\newpage
\exo[Tensor products in Dirac/Coecke notation]
\begin{mathpar}
\left(\begin{array}{c}
 1\\
 0
\end{array}\right)
\otimes
\left(\begin{array}{c}
 0\\
 1
\end{array}\right)
=
\left(\begin{array}{c}
 1\\
 0\\
 0\\
 0
\end{array}\right)
\and
\left(\begin{array}{c}
 0\\
 1
\end{array}\right)
\otimes
\left(\begin{array}{c}
 1\\
 0
\end{array}\right)
=
\left(\begin{array}{c}
 0\\
 0\\
 1\\
 0
\end{array}\right)
\and
\left(\begin{array}{c}
 1\\
 2
\end{array}\right)
\otimes
\left(\begin{array}{c}
 3\\
 4
\end{array}\right)
=
\left(\begin{array}{c}
 3\\
 4\\
 6\\
 8
\end{array}\right)\\
\left(\begin{array}{c}
 1\\
 0
\end{array}\right)
\otimes
\left(\begin{array}{c}
 1\\
 0
\end{array}\right)
+
\left(\begin{array}{c}
 0\\
 1
\end{array}\right)
\otimes
\left(\begin{array}{c}
 0\\
 1
\end{array}\right)
=
\left(\begin{array}{c}
 1\\
 0\\
 0\\
 0
\end{array}\right)
+
\left(\begin{array}{c}
 0\\
 0\\
 0\\
 1
\end{array}\right)
=
\left(\begin{array}{c}
 1\\
 0\\
 0\\
 1
\end{array}\right)
\end{mathpar}
The last two in Dirac notation :
\begin{align*}
  (\ket 0 + 2 \ket 1) \otimes (3 \ket 0 + 4 \ket 1)
  =& 3 \ket{00} + 4 \ket{01} + 6 \ket{10} + 8 \ket{11}
\end{align*}
\begin{align*}
  \ket 0 \otimes \ket 0 + \ket 1 \otimes \ket 1
  =& \ket{00} + \ket{11}
\end{align*}

\begin{mathpar}
\left(\begin{array}{cc}
 1&0\\
 0&1
\end{array}\right)
\otimes
\left(\begin{array}{cc}
 1&0\\
 0&1
\end{array}\right)
=
\left(\begin{array}{cccc}
  1&0&0&0\\
  0&1&0&0\\
  0&0&1&0\\
  0&0&0&1
\end{array}\right)
\and
\left(\begin{array}{cc}
 1&0\\
 0&0
\end{array}\right)
\otimes
\left(\begin{array}{cc}
 1&0\\
 0&1
\end{array}\right)
=
\left(\begin{array}{cccc}
  1&0&0&0\\
  0&1&0&0\\
  0&0&0&0\\
  0&0&0&0
\end{array}\right)
\\
\left(\begin{array}{cc}
 1/\sqrt{2}& 1/\sqrt{2}\\
 1/\sqrt{2}& -1/\sqrt{2}
\end{array}\right)
\otimes
\left(\begin{array}{cc}
 1/\sqrt{2}& 1/\sqrt{2}\\
 1/\sqrt{2}& -1/\sqrt{2}
\end{array}\right)
=
\left(\begin{array}{cccc}
  1/2& 1/2& 1/2& 1/2\\
  1/2&-1/2& 1/2&-1/2\\
  1/2& 1/2&-1/2&-1/2\\
  1/2&-1/2&-1/2& 1/2
\end{array}\right)
\end{mathpar}

In Dirac notation :

\begin{align*}
  &(\ket 0 \bra 0 + \ket 1 \bra 1) \otimes (\ket 0 \bra 0 + \ket 1 \bra 1) \\
  =& \ket 0 \bra 0 \otimes \ket 0 \bra 0 + \ket 0 \bra 0 \otimes \ket 1 \bra 1 +
  \ket 1 \bra 1 \otimes \ket 0 \bra 0 + \ket 1 \bra 1 \otimes \ket 1 \bra 1 \\
  =& \ket 0 \bra 0 + \ket 1 \bra 1 + \ket 2 \bra 2 +\ket 3 \bra 3
\end{align*}
\begin{align*}
  &(\ket 0 \bra 0) \otimes (\ket 0 \bra 0 + \ket 1 \bra 1) \\
  =& \ket 0 \bra 0 \otimes \ket 0 \bra 0 + \ket 0 \bra 0 \otimes \ket 1 \bra 1
  \\
  =& \ket 0 \bra 0 + \ket 1 \bra 1
\end{align*}
\begin{align*}
  1/\sqrt 2  (&\ket 0 \bra 0 + \ket 0 \bra 1 + \ket 1 \bra 0 - \ket 1 \bra 1 )
    \otimes 1/\sqrt 2 (\ket 0 \bra 0 + \ket 1 \bra 0 + \ket 1 \bra 0 - \ket 1
    \bra 1) \\
  =1/2 (&\ket 0 \bra 0 + \ket 0 \bra 1 + \ket 0 \bra 2 + \ket 0 \bra 3 + \\
   &\ket 1 \bra 0 - \ket 1 \bra 1 + \ket 1 \bra 2 - \ket 1 \bra 3 + \\
   &\ket 2 \bra 0 + \ket 2 \bra 1 - \ket 2 \bra 2 - \ket 2 \bra 3 + \\
   &\ket 3 \bra 0 - \ket 3 \bra 1 - \ket 3 \bra 2 + \ket 3 \bra 3)
\end{align*}

\newpage

We want to prove $(A \otimes B) (C \otimes D) = (AC) \otimes (BD)$

\begin{itemize}
  \item Dirac's notation :

  \item Coecke's notation :

\end{itemize}

\exo[Dagger in Dirac/Coecke notation]

\begin{align*}
\left(\begin{array}{cc}
 1/\sqrt{2}& 1/\sqrt{2}\\
 1/\sqrt{2}& -1/\sqrt{2}
\end{array}\right)^\dagger
=
\left(\begin{array}{cc}
 1/\sqrt{2}& 1/\sqrt{2}\\
 1/\sqrt{2}& -1/\sqrt{2}
\end{array}\right)
\qquad
\left(\begin{array}{cc}
 1& 3i\\
 2& 4i
\end{array}\right)^\dagger
=
\left(\begin{array}{cc}
 1& 2\\
 -3i& -4i
\end{array}\right)
\end{align*}
In Dirac notation:
\begin{align*}
  &1/\sqrt 2 (\ket 0 \bra 0 + \ket 0 \bra 1 + \ket 1 \bra 0 - \ket 1 \bra
    1)^\dagger \\
  =&1/\sqrt 2 (\ket 0 \bra 0 + \ket 1 \bra 0 + \ket 0 \bra 1 - \ket 1 \bra 1)
\end{align*}
\begin{align*}
  &(\ket 0 \bra 0 + 3i \ket 0 \bra 1 + 2 \ket 1 \bra 0 + 4i \ket 1 \bra
    1)^\dagger \\
  =&(\ket 0 \bra 0 - 3i \ket 1 \bra 0 + 2 \ket 0 \bra 1 - 4i \ket 1 \bra
    1)
\end{align*}

\exo[Gates in Dirac notations]

$$ H = 1/\sqrt 2 (\ket 0 \bra 0+\ket 0 \bra 1+\ket 1 \bra 0-\ket 1 \bra 1)$$
$$ \textit{CNot} = \ket 0 \bra 0+\ket 1 \bra 1+\ket 3 \bra 2+\ket 2 \bra 3$$
$$ T = \ket 0 \bra 0 + e^{\frac{i\pi}{4}} \ket 1 \bra 1$$

\exo[Pauli matrices in Dirac/Coecke notation]
.

\section{Postulates on pure states}

\exo[Evolutions]~

Let $\ket \psi = \ket{0} \otimes \ket{0}$ the initial state of two
qubits. We want to compute $\textit{CNot} (H \otimes I) \ket{\psi}$.

\begin{align*}
  \CNot (H \otimes I) (\ket 0 \otimes \ket 0) &=
    \CNot ((\ket 0 + \ket 1)/\sqrt 2 \otimes (\ket 0 + \ket 1)/\sqrt 2) \\
    &= \frac 1 {\sqrt 2} \CNot (\ket{00} + \ket{01} + \ket{10} + \ket{11}) \\
    &= \frac 1 {\sqrt 2} (\ket{00} + \ket{01} + \ket{11} + \ket{10}) := \ket{\psi'}
\end{align*}

\begin{align*}
  (H\otimes I)\CNot \ket{\psi'}
  &= (H\otimes I) \frac 1 {\sqrt 2} (\ket{00} + \ket{01} + \ket{10} + \ket{11}) \\
  &= \ket {00} & H^2\ket \psi = \ket \psi
\end{align*}

\begin{enumerate}
  \item $T^4H\; \ket 0$
  \item $HT^4H\; \ket 0$
  \item $\CNot(H\ket 0 \otimes HT^4H \ket 0)$
  \item 
  % \item $(I \otimes \CNot)
  %   (\CNot (H\ket 0 \otimes \ket 0) \otimes \ket 0)$
  \item $(H \otimes (H^2)) (\ket 0 \otimes \ket 0))$
\end{enumerate}

\exo[Measuring in another basis]~

\begin{itemize}
  \item orthogonal :
    \begin{align*}
      \braket + - &= (\frac 1 {\sqrt 2})^2 + \frac 1 {\sqrt 2} \times
          \frac{-1}{\sqrt 2} \\
            &= (\frac 1 {\sqrt 2})^2 - (\frac 1 {\sqrt 2})^2 \\
            &= 0
    \end{align*}

  \item norm one :
    \begin{multicols}{2}
      \begin{align*}
        \braket + + &= (\frac 1 {\sqrt 2})^2 + (\frac 1 {\sqrt 2})^2 \\
            &= \frac 1 2 + \frac 1 2 \\
            &= 1
      \end{align*}

      \begin{align*}
        \braket - - &= (\frac 1 {\sqrt 2})^2 + (\frac{-1}{\sqrt 2})^2 \\
            &= \frac 1 2 + \frac 1 2 \\
            &= 1
      \end{align*}
    \end{multicols}

  \item generate $\mathbb C^2$ 

    Let $c = \alpha \ket 0 + \beta \ket 1 \in \mathbb C^2$ we want to find $x$
    and $y$ such that $x \ket + + y \ket - = c$.

    \begin{align*}
      x \ket + + y \ket - &= 
      \frac x {\sqrt 2}(\ket 0 + \ket 1) + \frac y {\sqrt 2}(\ket 0 - \ket 1) \\
      &= (\frac {x + y} {\sqrt 2}) \ket 0 + (\frac {x - y}{\sqrt 2}) \ket 1
    \end{align*}

    So we have this system :
    \begin{align*}
      \begin{cases}
          (x + y) / \sqrt 2 &= \alpha \\
          (x - y) / \sqrt 2 &= \beta
        \end{cases}
      \Rightarrow& \begin{cases}
          x + y &= \sqrt 2 \alpha \\
          x - y &= \sqrt 2 \beta
        \end{cases} \\
      \Rightarrow& \begin{cases}
          2x &= \sqrt 2 (\alpha + \beta) \\
          x - y &= \sqrt 2 \beta
        \end{cases} \\
      \Rightarrow& \begin{cases}
          x &= \frac {\sqrt 2} 2 (\alpha + \beta) \\
          -x + y &= -\sqrt 2 \beta
        \end{cases} \\
      \Rightarrow& \begin{cases}
          x &= \frac {\sqrt 2} 2 (\alpha + \beta) \\
          y &= \frac {\sqrt 2} 2 (\alpha - \beta)
        \end{cases}
      \end{align*}
\end{itemize}

$B = \{\ket 0, \ket 1\}$ is another o.n.b of $\mathbb C^2$

We need to show that $\sum_{M\in\mathcal M_{\pm}} M^\dagger M = 1$
\begin{align*}
  \sum_{M\in\mathcal M_{\pm}} 
  &= (\ket + \bra +)^\dagger (\ket + \bra +) + 
     (\ket - \bra -)^\dagger (\ket - \bra -) \\
  &= (\ket + \braket + + \bra +) + (\ket - \braket - - \bra -)\\
  &= \ket + \bra + + \ket - \bra - \\
  &= \frac 1 2 (\ket 0 \bra 0 + \ket 0 \bra 1 + \ket 1 \bra 0 + \ket 1 \bra 1) +
     \frac 1 2 (\ket 0 \bra 0 - \ket 0 \bra 1 - \ket 1 \bra 0 + \ket 1 \bra 1)
     \\
  &= \ket 0 \bra 0 + \ket 1 \bra 1 \\
  &= 1
\end{align*}

So $\mathcal M_\pm$ is a valid measurement.

We have $\ket \psi = \frac 1 3 \ket 0 + \frac{\sqrt 8} 3 \ket 1$

\begin{itemize}
  \item For $\ket + \bra +$
    \begin{align*}
      p(\ket +) &= \bra \psi (\ket + \bra +)^\dagger \ket + \bra + \ket \psi \\
      &= \bra \psi \ket + \bra + \ket \psi \\
      &= \braket \psi + \braket + \psi \\
      &= (\frac 1 3 \times \frac 1 {\sqrt 2} + \frac {\sqrt 8} 3 \times \frac 1
        {\sqrt 2})^2 \\
      &= \frac 1 2 \times (\frac{1 + \sqrt 8} 3)^2
      = \frac 1 2 \times \frac {(1 + \sqrt 8)^2} 9
      = \frac {(1 + \sqrt 8)^2}{18}
    \end{align*}
  For $\ket - \bra -$
\end{itemize}

\exo[Measuring to distinguish]~

a

\section{Some mathematics}

\exo[Spectral theorems complements]~

Let $D$ and $D'$ two diagonal matrices and $U$ a unitary matrix.

\begin{itemize}
  \item Let $A = UDU^\dagger$ and $B=UD'U^\dagger$
    \begin{align*}
      AB &= UDU^\dagger UD'U^\dagger \\
        &= UDD'U^\dagger & \text{$U$ is a unitary matrix} \\
        &= UD'DU^\dagger & \text{$D$ and $D'$ are diagonal} \\
        &= UD'U^\dagger UDU^\dagger \\
        &= BA
    \end{align*}

  \item Let $M = UDU^\dagger$
\end{itemize}

\section{On the nature of quantum information}

\section{Protocols}

\section{Quantum error correction}

\section{Bell}

\end{document}
