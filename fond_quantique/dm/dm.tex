\documentclass[12pt]{article}
\usepackage{textcomp,amssymb,amsmath,amsthm,stmaryrd}
\usepackage{bold-extra}
\usepackage[utf8]{inputenc}
\usepackage[english] {babel}
\usepackage[pdftex]{graphicx,color}
\usepackage{environ}
\usepackage{verbatim}
\usepackage{enumerate}
\usepackage{multicol}
\usepackage{hyperref}
\hypersetup{colorlinks=true,linkcolor=blue}
\usepackage{fancyhdr}
\usepackage{mathpartir}

\usepackage[a4paper, headheight=10mm, hmargin={2cm, 2cm}, vmargin={2cm, 2cm}]{geometry}

\newcommand{\x}{\noindent{\color{blue}xxx}\\}
\newcommand{\xx}[1]{\noindent{\color{blue}#1}\\}
\newcommand{\xxx}[2]{{\color{red}  $<$#2$>$ #1 $<\backslash $#2$>$ }}
\newcommand{\couic}[1]{}
\newcommand{\couicfootnote}[1]{}
\newcommand{\couicefootnote}[1]{}
\newcommand{\couicfootnotemark}{}

%\newcommand{i.e.}{{\em i.e. }}
%\newcommand{e.g.}{{\em e.g. }}
%\newcommand{etc.}{{\em etc. }}
%\newcommand{c.f.}{{\em cf. }}
\newcommand{\eg}{e.g.}
\newcommand{\etc}{etc.}
\newcommand{\ie}{i.e.}

\newcommand{\lra}{\longrightarrow}
\newcommand{\ul}{\underline}
\newcommand{\sub}[1]{\langle #1 \rangle}
\newcommand{\ve}[1]{\mathrm{\textbf{#1}}}

\newcommand{\ket}[1]{| #1 \rangle}
\newcommand{\bra}[1]{\langle #1 |}
\newcommand{\braket}[2]{\langle #1 | #2 \rangle}
\newcommand{\trace}{\textrm{Tr}}

\newcommand{\dom}{\textrm{dom}}
\newcommand{\range}{\textrm{range}}

\newcommand{\Herm}{\textrm{Herm}^+_{d^2}(\mathbb{C})}
\newcommand{\herm}{\textrm{Herm}^+_{d}(\mathbb{C})}
\newcommand{\hermdm}{\textrm{Herm}^+_{dm}(\mathbb{C})}
\newcommand{\hermm}{\textrm{Herm}^+_{m}(\mathbb{C})}

\newcommand{\im}{\mathrm Im}
\newcommand{\ii}{\mathrm i}
\newcommand{\europ}{\textup{\emph{\geneuro}}}
\newcommand{\ad}{\operatorname{ad}}
\newcommand{\Id}{\operatorname{Id}}
\newcommand{\C}{\mathcal{C}}
\newcommand{\Z}{\mathbb{Z}}
\newcommand{\R}{\mathbb{R}}
\newcommand{\Supp}{{\bf\mathcal{S}}}
\newcommand{\pa}[1]{\left(#1\right)}
\newcommand{\acco}[1]{\left\{#1\right\}}
\newcommand{\norme}[1]{\left|\left|#1\right|\right|}

\newcommand{\tzero}{{\overline{0}}}
\newcommand{\size}{\textrm{\texttt{size}}}
\newcommand{\corch}[1]{[#1]}
\newcommand{\val}[1]{[\![#1]\!]}
\newcommand{\llaves}[1]{\{#1\}}
\newcommand{\true}{\mathrm{true}}
\newcommand{\false}{\mathrm{false}}
\newcommand{\type}{\colon\!}
\newcommand{\T}{\mathcal{T}}
\newcommand{\U}{\mathcal{U}}
\newcommand{\Sc}{\mathcal{S}}
\newcommand{\B}{\mathcal{B}}
\newcommand{\LL}{\mathcal{LL}}
\newcommand{\BLL}{\mathcal{BLL}}
\newcommand{\SL}{\ensuremath{\mathcal{SL}}}
\newcommand{\mapLLSL}[1]{\kappa_U({#1})}
\newcommand{\lasf}{\lambda 2^{la}}
\newcommand{\rulesf}{^\triangleleft}
\newcommand{\mapsf}{^\natural}
\newcommand{\Tsf}{\mathbb{T}(\lasf)}


\newcommand{\CNot}{\textit{CNot}}
\newcommand{\CSwap}{\textit{CSwap}}

\setlength{\topmargin} {-2.5cm}
\setlength{\oddsidemargin} {-1.1cm}
\setlength{\textwidth} {19cm}
\setlength{\textheight} {25cm}
\setlength{\parindent}{0pt}
\newcommand{\compresslist}{
\setlength{\topsep}{0pt}
\setlength{\itemsep}{0pt}
\setlength{\itemindent}{2em}
}

%Pour version avec corrections, décommenter les 3 lignes ci-dessous et commenter le %

%\NewEnviron
%\newenvironment{correct}
%{\color{red}}
%{}

% Pour version sans corrections à donner aux étudiants, commenter les 3 lignes ci-dessus et décommenter celle ci-dessous
\NewEnviron{correct}{}

\newtheoremstyle{exostyle}
  {0.2cm}{0.5cm}%                                 margin top and bottom
  {\rmfamily}%                                  text layout
  {0cm}%                                        indention of header
  {\bfseries}{ }%                               header font and text after
  {0cm}%                                        space after header
  {\thmname{#1}\thmnumber{ #2}:\thmnote{ #3}}%  header

\theoremstyle{exostyle}
 \newtheorem{exo}{Exercice}[section]
\renewcommand{\marks}[1]{{\small \color{magenta} \noindent \textbf{(#1 point(s) $\Box$)}
}}

\newcommand{\sem}[1]{\left\llbracket#1\right\rrbracket}
\newcommand{\cmp}[1]{\overline{#1}}

\lhead{Foundations of Quantum Information}
\chead{Valeran MAYTIE}
\rhead{M1 MPRI T2}

\pagestyle{fancy}

\begin{document}

\begin{center}
{\bf {\Large  Homework}}\\
\end{center}

\section{Basic operation and their notation}

\exo[Inner/outer products in Dirac notation]

\begin{align*}
\left(\begin{array}{cccc}
 1 &0 \\
\end{array}\right)
\left(\begin{array}{c}
 1\\
 0 \\
\end{array}\right) =
\left(\begin{array}{c}
 1
\end{array}\right)
\quad
\left(\begin{array}{cccc}
 1 &0 \\
\end{array}\right)
\left(\begin{array}{c}
 0\\
 1
\end{array}\right) =
\left(\begin{array}{c}
 0\\
\end{array}\right)
\quad
\left(\begin{array}{cccc}
 1 &2 \\
\end{array}\right)
\left(\begin{array}{c}
 3\\
 4
\end{array}\right) =
\left(\begin{array}{c}
 11
\end{array}\right)
\end{align*}
The last one in Dirac notation :
\begin{align*}
  & (\bra{0} + 2\bra{1}) \times (3\ket 0 + 4 \ket 1) \\
  =& 3\braket 0 0 + 4 \braket 0 1 + 6 \braket 1 0 + 8 \braket 1 1 \\
  =& 3 + 8 \\
  =& 11
\end{align*}
\begin{align*}
\left(\begin{array}{c}
 1\\
 0
\end{array}\right)
\left(\begin{array}{cccc}
 1 &0 \\
\end{array}\right)
=
\left(\begin{array}{cccc}
 1 &0 \\
 0 &0 \\
\end{array}\right)
\qquad
\left(\begin{array}{c}
 0\\
 1
\end{array}\right)
\left(\begin{array}{cccc}
 1 &0 \\
\end{array}\right)
=
\left(\begin{array}{cccc}
 0 &0 \\
 1 &0 \\
\end{array}\right)
\qquad
\left(\begin{array}{c}
 3\\
 4
\end{array}\right)
\left(\begin{array}{cccc}
 1 &2 \\
\end{array}\right)
=
\left(\begin{array}{cccc}
 3 &6 \\
 4 &8 \\
\end{array}\right)
\end{align*}
The last one in Dirac notation :
\begin{align*}
  & (3\ket{0} + 4\ket{1}) \times (\bra 0 + 2\bra 1) \\
  =& 3 \ket 0 \bra 0 + 6 \ket 0 \bra 1 + 4 \ket 1 \bra 0 + 8 \ket 1 \bra 1
\end{align*}

\exo[Matrix products in Dirac notation]
\begin{align*}
\left(\begin{array}{cccc}
 0 &0 \\
 1 &0
\end{array}\right)
\left(\begin{array}{c}
 1\\
 0
\end{array}\right)
=
\left(\begin{array}{c}
 0\\
 1
\end{array}\right)
\qquad
\left(\begin{array}{cccc}
 0 &0\\
 1 &0
\end{array}\right)
\left(\begin{array}{c}
 0\\
 1
\end{array}\right)
=
\left(\begin{array}{c}
 0\\
 0
\end{array}\right)
\qquad
\left(\begin{array}{cccc}
 1 &3 \\
 2 &4
\end{array}\right)
\left(\begin{array}{c}
 5\\
 6
\end{array}\right)
=
\left(\begin{array}{c}
 23\\
 34
\end{array}\right)
\end{align*}
The last one in Dirac notation :
\begin{align*}
  &(\ket 0 \bra 0 + 3\ket 0 \bra 1 + 2 \ket 1 \bra 0 + 4 \ket 1 \bra 1)\times
  (5\ket 0 + 6 \ket 1) \\
  =&5\ket 0 \braket 0 0 + 6 \ket 0 \braket 0 1 + 15 \ket 0 \braket 1 0 + 18
    \ket 0 \braket 1 1 + \\
  & 10 \ket 1 \braket 0 0 + 12 \ket 1 \braket 0 1 + 20 \ket 1 \braket 1 0 +
    24 \ket 1 \braket 1 1\\
  =&5 \braket 0 0 \ket 0 + 18 \braket 1 1 \ket 0 + 10 \braket 0 0 \ket 1 +
  24 \braket 1 1 \ket 1 \\
  =&5 \ket 0 + 18 \ket 0 + 10 \ket 1 + 24 \ket 1 \\
  =&23  \ket 0 + 24 \ket 1
\end{align*}

\begin{mathpar}
\left(\begin{array}{cc}
 1& 0\\
 0& 1
\end{array}\right)
\left(\begin{array}{cc}
 1& 3\\
 2& 4
\end{array}\right)
=
\left(\begin{array}{cc}
 1& 3\\
 2& 4
\end{array}\right)
\and
\left(\begin{array}{cc}
 0& 0\\
 0& 1
\end{array}\right)
\left(\begin{array}{cc}
 0& 0\\
 0& 1
\end{array}\right)
=
\left(\begin{array}{cc}
 0& 0\\
 0& 1
\end{array}\right)
\newline
\left(\begin{array}{cc}
 1/\sqrt{2}& 1/\sqrt{2}\\
 1/\sqrt{2}& -1/\sqrt{2}
\end{array}\right)
\left(\begin{array}{cc}
 1/\sqrt{2}& 1/\sqrt{2}\\
 1/\sqrt{2}& -1/\sqrt{2}
\end{array}\right)
=
\left(\begin{array}{cc}
  1& 0\\
  0& 1
\end{array}\right)
\end{mathpar}
The last one in Dirac notation :
\begin{align*}
  &(1/\sqrt 2\ket 0 \bra 0 + 1/\sqrt 2\ket 0 \bra 1 + 1/\sqrt 2 \ket 1 \bra 0
  -1/\sqrt 2 \ket 1 \bra 1)^2 \\
  =& 1/2 \ket 0 \braket 0 0 \bra 0 + 1/2 \ket 0 \braket 0 0 \bra 1 + 1 /2 \ket 0
  \braket 1 1 \bra 0 - 1 / 2 \ket 0 \braket 1 1 \bra 1 \\
  & 1/2 \ket 1 \braket 0 0 \bra 1 + 1/2 \ket 1 \braket 0 0 \bra 0 - 1 /2 \ket 1
  \braket 1 1 \bra 0 + 1 /2 \ket 1 \braket 1 1 \bra 1\\
  =& 1/2 \ket 0 \bra 0 + 1 /2 \ket 0 \bra 0 +
  1/2 \ket 1 \bra 1 + 1 /2 \ket 1 \bra 1\\
  =& \ket 0 \bra 0 + \ket 1 \bra 1
\end{align*}

\end{document}
