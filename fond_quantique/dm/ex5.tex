\section{Protocols}

\exo[Canonical basis versus diagonal basis]~

\begin{itemize}
  \item If Bob measures the result in the same basis then he can retrieve the
    information sent

  \item If Bob measures in an other basis then he learns nothing about the
    message.

  \item If Eve intercepts and measures it in the same basis then Bob can have
    some information on the message if he read the message in the same basis.

  \item But if Eve intercepts and measures it in an other basis and Bob read the
    message in the original basis then he learns nothing about the message.
\end{itemize}

\exo[BB84]~

\begin{enumerate}
  \item Alice will start by producing a random string of bits, encode each of
    them either into the canonical or the diagonal basis, and send that to Bob.

  \item Bob will measure them either using the canonical basis or the
    diagonal basis, at random.

  \item Bob will broadcast which bases he used

  \item Alice will know when Bob used the same base. When Bob has used the right
    base, Bob's information is correct, otherwise it is wrong (previous exercise).

  \item Eve does not know the bases like Bob And she has very little chance of
    having the right basic sequence ($\frac{1}{2^n}$). But she's going to
    disrupt Bob's measurements.

  \item 
\end{enumerate}

They can use common measurements bases to create an encryption key. For example,
a basic measurements base sequence can become a binary code with 0 when we have
the base $\mathcal M$ and 1 if we have the base $\mathcal M'$. With this
generate key we can communicate with an existing encryption protocol.

\exo[Quantum random access code]~

We have 4 cases (The calculations are not too detailed to make them easier to read):

\begin{itemize}
  \item $x = y = 0$
    \begin{align*}
      p(\ket +) &= \frac 1 2 + \frac 1 {2\sqrt 2} &
      p(\ket{-i}) &= \frac 1 2 + \frac 1 {2\sqrt 2}
    \end{align*}
  \item $x = 0$ and $y = 1$
    \begin{align*}
      p(\ket +) &= \frac 1 2 + \frac 1 {2\sqrt 2} &
      p(\ket{i}) &= \frac 1 2 + \frac 1 {2\sqrt 2}
    \end{align*}

  \item $x = 1$ and $y = 0$
    \begin{align*}
      p(\ket -) &= \frac 1 2 + \frac 1 {2\sqrt 2} &
      p(\ket{-i}) &= \frac 1 2 + \frac 1 {2\sqrt 2}
    \end{align*}

  \item $x = y = 1$
    \begin{align*}
      p(\ket -) &= \frac 1 2 + \frac 1 {2\sqrt 2} &
      p(\ket{-i}) &= \frac 1 2 + \frac 1 {2\sqrt 2}
    \end{align*}
\end{itemize}

If Bob wants to find $x$ he needs to use $\mathcal M_x$.
If Bob wants to find $y$ he needs to use $\mathcal M_{xi}$.

With this method the probability to success is : 

\begin{align*}
  \mathbb P(win) &= \frac 1 4(p(+)_{x=1} + p(-)_{x=0} + p(+i)_{y=0} + p(-i)_{y=1})
  \\
  &= \frac 1 4 \times 4 (\frac 1 2 + \frac 1 { 2 \sqrt 2}) \\
  &= \frac 1 2 + \frac 1 { 2 \sqrt 2} \\
  &> \frac 3 4
\end{align*}

\exo[The Bell basis]~

\begin{align*}
  \ket{\beta_0} &= \frac{1}{\sqrt 2}(\ket{00} + \ket{11})
  &\ket{\beta_1} &= (X \otimes I)\ket{\beta_0} =
    \frac{1}{\sqrt 2}(\ket{10}+\ket{01}) \\
  \ket{\beta_2} &= (Y \otimes I)\ket{\beta_2} =
    \frac{i}{\sqrt 2}(\ket{10} - \ket{01})
  &\ket{\beta_3} &= (Z \otimes I)\ket{\beta_0} =
    \frac{1}{\sqrt 2}(\ket{00}-\ket{11})
\end{align*}

These four states are orthogonal and orthonormal,

orthogonal:
\begin{align*}
  \braket{\beta_0}{\beta_1} &= \frac 1 2 \times 0 = 0 &
  \braket{\beta_0}{\beta_2} &= \frac i 2 \times 0 = 0 \\
  \braket{\beta_0}{\beta_3} &= \frac 1 2 - \frac 1 2 = 0 &
  \braket{\beta_1}{\beta_2} &= \frac i 2 - \frac i 2 = 0 \\
  \braket{\beta_1}{\beta_3} &= \frac 1 2 \times 0 = 0 &
  \braket{\beta_2}{\beta_3} &= \frac i 2 \times 0 = 0
\end{align*}

orthonormal:
\begin{align*}
  \braket{\beta_0}{\beta_0} &= \frac 1 2 + \frac 1 2 = 1 &
  \braket{\beta_1}{\beta_1} &= \frac 1 2 + \frac 1 2 = 1 \\
  \braket{\beta_2}{\beta_2} &= \frac 1 2 + \frac 1 2 = 1 &
  \braket{\beta_3}{\beta_3} &= \frac 1 2 + \frac 1 2 = 1 \\
\end{align*}

So, this states are an orthonormal basis.

It is also a valid measurement :

\begin{align*}
  \sum_i \mathcal M _i &=
  \ket {\beta_0} \bra {\beta_0} + \ket {\beta_1} \bra {\beta_1} +
  \ket {\beta_2} \bra {\beta_2} + \ket {\beta_3} \bra {\beta_3} \\
  &= I_4
\end{align*}

\exo[Superdense coding]~

At the beginning, Alice and Bob share an entangled state $\ket{\beta_0} = \frac
1 {\sqrt 2}(\ket{00} + \ket{11}$ Alice can change $\ket{\beta_0}$ in $\ket
{\beta_k}$ with this operation :

\begin{itemize}
  \item $\ket{\beta_0}$ : do nothing
  \item $\ket{\beta_1}$ : apply the matrix $X$
    \begin{align*}
      X\ket{\beta_0} &= (X \otimes I) \frac{1}{\sqrt 2}(\ket{00} + \ket{11}) \\
      &= \frac{1}{\sqrt 2}(\ket{10} + \ket{01}) \\
      &= \ket{\beta_1}
    \end{align*}
  \item $\ket{\beta_2}$ : apply the matrix $Y$
    \begin{align*}
      Y\ket{\beta_0} &= (Y \otimes I) \frac{1}{\sqrt 2}(\ket{00} + \ket{11}) \\
      &= \frac{i}{\sqrt 2}(\ket{10} - \ket{01}) \\
      &= \ket{\beta_2}
    \end{align*}
  \item $\ket{\beta_3}$ : apply the matrix $Z$
    \begin{align*}
      Y\ket{\beta_0} &= (Z \otimes I) \frac{1}{\sqrt 2}(\ket{00} + \ket{11}) \\
      &= \frac{1}{\sqrt 2}(\ket{00} - \ket{11}) \\
      &= \ket{\beta_3}
    \end{align*}
\end{itemize}

So Alice can encode the four possible pairs of bits (00, 01, 10 and 11) with the
4 Bell states. We have shown that Alice can modify $\ket{B_0}$ on her own, so
with  qubit she can change the communication state to one of the 4 states. Bob
can measure the result and retrieve the information from Alice.

\exo[Discussion: classical description of a single qubit]~

A qubit is coded with this formula : $\alpha \ket 0 + \beta \ket 1$.
We just need to send $2$ complexes numbers. So if a number is encoded with $n$
bits we send $4n$ bits.


\exo[Teleportation]~

\begin{align*}
  \frac 1 2 \sum_i \ket{\beta_i} \otimes \sigma_i \ket{\psi}
  =& \frac{1}{2\sqrt 2}((\ket{00}+\ket{11}) \otimes (\alpha \ket 0 + \beta \ket 1)+\\
  &(\ket{10}+\ket{01}) \otimes (\alpha \ket 1 + \beta \ket 0)+\\
  &(i\ket{10} - i \ket{01}) \otimes (i\alpha \ket 1 - i\beta \ket 0)+\\
  &(\ket{00} - \ket{11}) \otimes (\alpha \ket 0 - \beta \ket 1)) \\
  =& \frac{1}{\sqrt 2}(\alpha \ket{000} + \beta \ket{100} + \alpha \ket{011} +
  \beta \ket{111} \\
  =& \frac{1}{2\sqrt 2}(2\alpha(\ket{000} + \ket{011}) + 2\beta(\ket{100} +
  \ket{111})) \\
  =& \frac{1}{\sqrt 2}(\alpha(\ket{000} + \ket{011}) + \beta(\ket{100} + \ket{111}))
    \\
  =& \frac{1}{\sqrt 2}(
    (\alpha \ket{0} + \beta \ket{1}) \otimes \ket 0 \otimes \ket 0 + 
    (\alpha \ket{0} + \beta \ket{1}) \otimes \ket 1 \otimes \ket 1) \\
  =& \frac{1}{\sqrt 2}(\ket{\psi} \otimes \ket{0} \otimes \ket{0}) +
    (\ket{\psi} \otimes \ket 1 \otimes \ket 1)
\end{align*}

\exo[The swap test]~

Before the measurement we have this state :

\begin{align*}
  \ket{\kappa}&=(H \otimes I \otimes I) \CSwap (H \otimes I \otimes I) \ket{0} \otimes
  \ket{\phi} \otimes \ket{\psi} \\
  &= (H \otimes I \otimes I) \CSwap ((\frac{1}{\sqrt 2}(\ket{0} + \ket{1}))
  \otimes \ket{\phi} \otimes \ket{\psi}) \\
  &= (H \otimes I \otimes I) \CSwap \frac{1}{\sqrt 2}
     (\ket{0\phi\psi} + \ket{1\phi\psi}) \\
  &= (H \otimes I \otimes I)\frac{1}{\sqrt 2}
     (\ket{0\phi\psi} + \ket{1\psi\phi}) \\
  &= \frac 1 2(\ket 0 \otimes (\ket{\phi\psi} + \ket{\psi\phi}) -
               \ket 1 \otimes (\ket{\phi\psi} + \ket{\psi\phi})) \\
  &= \frac 1 2(\ket{0\phi\psi} + \ket{0\psi\phi} +
               \ket{1\phi\psi} - \ket{1\psi\phi})
\end{align*}

We apply the measures :

\begin{align*}
  p(0) &= \bra \kappa \ket 0 \bra 0 \ket \kappa \\
  &= \frac 1 2 (\bra{\phi\psi} + \ket{\psi\phi}) \times
     \frac 1 2 (\bra{\phi\psi} + \ket{\psi\phi}) \\
  &= \frac 1 4 (2 + \braket{\phi\psi}{\psi\phi} + \braket{\psi\phi}{\phi\psi})\\
  &= \frac 1 4 (2 + 2\braket{\phi\psi}{\psi\phi})\\
  &= \frac 1 2 + \frac{\braket{\psi\phi}{\psi\phi}}{2}\\
  &= \frac 1 2 + \frac 1 2 |\braket{\psi}{\phi}|^2\\
\end{align*}

we have $p(1) = 1 - p(0)$ so $p(1) = \frac 1 2 - \frac 1 2
|\braket{\psi}{\phi}|^2$

\exo[Quantum fingerprinting]~

TODO


