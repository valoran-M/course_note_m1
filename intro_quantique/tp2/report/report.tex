\documentclass{article}

\usepackage[utf8]{inputenc}
% \usepackage{braket}
\usepackage{enumitem}
\usepackage{multirow}
\usepackage{xcolor}
\usepackage[T1]{fontenc}
% \usepackage[french]{babel}
\usepackage{amssymb}
\usepackage{mathtools}
\usepackage{ntheorem}
\usepackage{amsmath}
\usepackage{amssymb}
\usepackage[ a4paper, hmargin={2cm, 2cm}, vmargin={3cm, 3cm}]{geometry}
\usepackage{hyperref}
\usepackage{capt-of}

\usepackage[braket, qm]{qcircuit}
\usepackage{graphicx}

\usepackage{tikz}
\usetikzlibrary{angles,quotes}

\theoremstyle{plain}
\theorembodyfont{\normalfont}
\theoremseparator{~--}
\newtheorem{exo}{Exercise}%[section]

\newcommand{\norm}[1]{\left\lVert#1\right\rVert}

\usepackage{hyperref}
\hypersetup{
    colorlinks,
    citecolor=black,
    filecolor=black,
    linkcolor=blue,
    urlcolor=blue
}

\usepackage{xcolor}

\definecolor{codegreen}{rgb}{0,0.6,0}
\definecolor{codegray}{rgb}{0.5,0.5,0.5}
\definecolor{codepurple}{rgb}{0.58,0,0.82}

\usepackage{listings}
\lstdefinestyle{mystyle}{
    commentstyle=\color{codegreen},
    keywordstyle=\color{magenta},
    numberstyle=\tiny\color{codegray},
    stringstyle=\color{codepurple},
    basicstyle=\ttfamily\footnotesize,
    breakatwhitespace=false,
    breaklines=true,
    captionpos=b,
    keepspaces=true,
    numbers=left,
    numbersep=5pt,
    showspaces=false,
    showstringspaces=false,
    showtabs=false,
    tabsize=2
}
\lstset{style=mystyle}

\title{VQE and QAOA to solve MAXCUT}
\author{Valeran MAYTIE}
\date{}

\begin{document}
  \maketitle

  \section{Coding graphs}

    \subsection{Cuts in graphs}

      We need to cut every links from a vertex in $V_0$ to a vertex in $V_1$\\

      \begin{lstlisting}[language=python, label=code:cuts,
                     caption=Code that computes the cost of a cut $s$ in a graph
                     $g$]
def costOfCut(g,s):
    r = 0
    l = len(s) - 1
    for n, e in g.items():
        for d in e:
            if s[l - d] != s[l - n]:
                r += 1
    return r\end{lstlisting}

    \subsection{Scalar product}

      \begin{lstlisting}[language=python, label=code:cuts,
                     caption=Code that computes the scalar product who whant to
                     optimize]
def scalprod(g,d):
    r = 0
    for k, p in d.items():
        r += costOfCut(g, k) * p
    return -r\end{lstlisting}

  \section{MAXCUT with VQE}

    \begin{lstlisting}[language=python, label=code:VQE,
                     caption=code that creates and simulates the VQE circuit]
def probDistVQE(a):
    q = QuantumRegister(4)
    c = ClassicalRegister(4)
    qc = QuantumCircuit(q,c)

    assert(len(a) % 12 == 0)

    for i in range(len(a) // 12):
        for qi in range(4):
            j = qi * 3 + i * 12
            qc.u(a[j], a[j + 1], a[j + 2], q[qi])
        for qi in range(4):
            qc.cnot(q[qi], q[(qi+1) % 4])  
    qc.measure(q, c)
    
    backend = BasicAer.get_backend('qasm_simulator')
    job = execute(qc, backend, shots=1000)
    res = dict(job.result().get_counts(qc))
    
    for i in res:
        res[i] = res[i] / 1000
    return res\end{lstlisting}

  \begin{itemize}
    \item What are the various proposed cuts $(V_{0},V_{1})$ ?

      \begin{center}
      \begin{tabular}{c|l}
        cuts & probabilities \\ \hline
        1001 & 0.699  \\
        0110 & 0.292 \\
        1010 & 0.003 \\
        0011 & 0.005 \\
        1100 & 0.001 \\

      \end{tabular}
      \end{center}

    \item Change the graph and check that it does not work ``just by chance''
      (for instance, use $g_2$, $g_3$, or your own).

      It also work for $g_2$ and $g_3$.

    \item Are all possible cuts there ?

    \item The problem is symmetric, in the sense that if '0110' is an answer,
      so is '1001'. Is this reflected in the resulting probabilities ?

      There's symmetry, but you don't get it every time

    \item If we had access to a real quantum co-processor, how would the
      code change ?

  \end{itemize}

  \section{MAXCUT with QAOA}

  \begin{lstlisting}[language=python, label=code:VQE,
                     caption=code that add V gate in a quantum circuit]
def V(qc,q,angle,g):
    for s in g:
        for e in g[s]:
            qc.cnot(q[s], q[e])
            qc.rz(angle, q[e])
            qc.cnot(q[s], q[e])
    return\end{lstlisting}

    If we exectue V on an empty circuit with graph $g_1$ and angle $ = 1.234$ we
    have this circuit :

    \begin{center}
    \scalebox{0.8}{
\Qcircuit @C=1.0em @R=0.2em @!R { \\
	 	\nghost{{q}_{0} :  } & \lstick{{q}_{0} :  } & \ctrl{1} & \qw & \ctrl{1} & \ctrl{2} & \qw & \ctrl{2} & \qw & \qw & \qw & \qw & \qw & \qw & \qw & \qw & \qw & \qw & \qw\\
	 	\nghost{{q}_{1} :  } & \lstick{{q}_{1} :  } & \targ & \gate{\mathrm{R_Z}\,(\mathrm{1.234})} & \targ & \qw & \qw & \qw & \ctrl{1} & \qw & \ctrl{1} & \ctrl{2} & \qw & \ctrl{2} & \qw & \qw & \qw & \qw & \qw\\
	 	\nghost{{q}_{2} :  } & \lstick{{q}_{2} :  } & \qw & \qw & \qw & \targ & \gate{\mathrm{R_Z}\,(\mathrm{1.234})} & \targ & \targ & \gate{\mathrm{R_Z}\,(\mathrm{1.234})} & \targ & \qw & \qw & \qw & \ctrl{1} & \qw & \ctrl{1} & \qw & \qw\\
	 	\nghost{{q}_{3} :  } & \lstick{{q}_{3} :  } & \qw & \qw & \qw & \qw & \qw & \qw & \qw & \qw & \qw & \targ & \gate{\mathrm{R_Z}\,(\mathrm{1.234})} & \targ & \targ & \gate{\mathrm{R_Z}\,(\mathrm{1.234})} & \targ & \qw & \qw\\
\\ }}
      \end{center}

    \newpage
  \begin{lstlisting}[language=python, label=code:QAOA,
                     caption=code that creates and simulates the QAOA circuit]
def probDistQAOA(a,g):
    p = int(len(a)/2)
    beta = a[:p]
    gamma = a[p:]
    
    q = QuantumRegister(4)
    c = ClassicalRegister(4)
    qc = QuantumCircuit(q,c)
    
    
    qc.h(q)
    
    
    for i in range(p):
        V(qc,q,gamma[i],g)
        for qb in q:
            qc.rx(beta[i], qb)    
    qc.measure(q, c)
    
    backend = BasicAer.get_backend('qasm_simulator')
    job = execute(qc, backend, shots=1000)
    res = dict(job.result().get_counts(qc))
    
    for i in res:
        res[i] = res[i] / 1000
    return res\end{lstlisting}


  \begin{itemize}
    \item Does it still work with other graphs ?
      
      Yes, it still work.

    \item Does it find all of the possibilities ? How about the symmetry of
      the results ?

      The symmetry is much more common than VQE.

    \item Play with the length of the array $a$ ($p=2,4,...$ -- make sure to
      keep it even). Do you see any loss/increase in precision ?

      Yes if we increase the length of the array $a$ we can an improvement of
      the precision but the exectution is slower.


    \item Remark how the number of necessary parameters is way smaller than for
      VQE.
  \end{itemize}


\end{document}

