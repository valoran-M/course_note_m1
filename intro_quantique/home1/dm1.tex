\documentclass{article}

\usepackage[utf8]{inputenc}
\usepackage{braket}
\usepackage{enumitem}
\usepackage{multirow}
\usepackage{xcolor}
\usepackage[T1]{fontenc}
% \usepackage[french]{babel}
\usepackage{amssymb}
\usepackage{mathtools}
\usepackage{ntheorem}
\usepackage{amsmath}
\usepackage{amssymb}
\usepackage[ a4paper, hmargin={3cm, 3cm}, vmargin={3cm, 3cm}]{geometry}
\usepackage{hyperref}
\usepackage{capt-of}

\usepackage{tikz}
\usetikzlibrary{angles,quotes}

\theoremstyle{plain}
\theorembodyfont{\normalfont}
\theoremseparator{~--}
\newtheorem{exo}{Exercise}%[section]

\newcommand{\norm}[1]{\left\lVert#1\right\rVert}

\usepackage{hyperref}
\hypersetup{
    colorlinks,
    citecolor=black,
    filecolor=black,
    linkcolor=blue,
    urlcolor=blue
}

\title{Home work 1}
\author{Valeran MAYTIE}
\date{}

\begin{document}
  \maketitle

  \section{Bell Basis}

    \begin{align*}
      \ket{\Phi^+} &= \frac{1}{\sqrt 2} (\ket{00} + \ket{11}) \\
      \ket{\Phi^-} &= \frac{1}{\sqrt 2} (\ket{00} - \ket{11}) \\
      \ket{\Psi^+} &= \frac{1}{\sqrt 2} (\ket{01} + \ket{10}) \\
      \ket{\Psi^+} &= \frac{1}{\sqrt 2} (\ket{01} - \ket{10})
    \end{align*}

    \begin{exo}
      Show that these 4 vectors form an orthonormal basis

      \begin{enumerate}
        \item We need to show that the norm of the 4 vectors is equal to 1

          This part is simple, because they all have the same constant 2 times,
          except for a sign which is ignored because it is squared.

          $$ \frac{1}{\sqrt 2}^2 + (\mp\frac{1}{\sqrt 2})^2
             = \frac 1 2 + \frac 1 2 = 1$$

        \item We need to show that they are pairwise orthogonal

          It simple to show that $\ket{\Phi^\mp}$ and $\ket{\Psi^\mp}$ are
          orthogonal, because the constants are not on the same ``ket'', so we
          have $\langle \Phi^\mp|\Psi^\mp \rangle =
          \frac{1}{\sqrt{2}} \times 0 
          +   0 \times \frac{1}{\sqrt{2}}
          \mp 0 \times \frac{1}{\sqrt{2}}
          \mp \frac{1}{\sqrt{2}} \times 0 = 0$

          Finally, the same calculation is used to show $\ket{\Phi^+}$ is
          orthogonal to $\ket{\Phi^-}$ and $\ket{\Psi^+}$ is orthogonal to
          $\ket{\Psi^-}$.

          \begin{align*}
            \langle \Phi^+ | \Phi^- \rangle = \langle \Psi^+ | \Psi^- \rangle
            = \frac{1}{\sqrt{2}} \times \frac{1}{\sqrt{2}} +
              \frac{1}{\sqrt{2}} \times \frac{-1}{\sqrt{2}}
            = \frac 1 2 - \frac 1 2 = 0
          \end{align*}
      \end{enumerate}

      All norm of the vectors is equal to 1 and they are all orthogonal.

      These 4 vectors form an orthonormal basis.
    \end{exo}

    \begin{exo}
      Consider the ket vector

      $$\ket{\varphi} = \frac{2}{\sqrt 5} \ket{01} + \frac{i}{\sqrt 5} \ket{10}$$

      \begin{enumerate}[label=(\alph*)]
        \item Show that this is a vector of norm 1

          $$ \langle \varphi|\varphi \rangle =
             \left( \frac{2}{\sqrt 5} \right) ^2 +
                    \frac{i}{\sqrt 5} \times \overline{\frac{i}{\sqrt 5}}
            = \frac 4 5 + \frac{i \times -i}{5} = \frac 4 5 + \frac 1 5 = 1$$

          \item Write it as a linear combination of $\ket{\Phi^+}, \ket{\Phi^-},
                \ket{\Psi^+}$ and $\ket{\Psi^-}$.

            We want to fin $\alpha$ and $\beta$ such that
            $\alpha \ket{\Psi^+} + \beta \ket{\Psi^-} = \ket{\varphi}$.
            We don't need $\ket{\Phi^\mp}$, because $\ket{00}$ and $\ket{11}$
            don't appear in $\ket{\varphi}$.
\newpage
$$
\left\{
\begin{array}{r l c}
  \alpha \frac{1}{\sqrt 2} + \beta \frac{1}{\sqrt 2} &= \frac{2}{\sqrt 5} \\
  \alpha \frac{1}{\sqrt 2} - \beta \frac{1}{\sqrt 2} &= \frac{i}{\sqrt 5}
\end{array}
\right. \Rightarrow
\left\{
\begin{array}{c c c l}
  \alpha \frac{1}{\sqrt 2} &+& \beta \frac{1}{\sqrt 2} &= \frac{2}{\sqrt 5} \\
  \alpha \frac{2}{\sqrt 2} & & &= \frac{2}{\sqrt 5} + \frac{i}{\sqrt 5} \Rightarrow
  \alpha = \frac{2 \sqrt 2}{2 \sqrt 5} + \frac{i \sqrt 2}{2 \sqrt 5} =
  \frac{\sqrt{10}}{5} + \frac{i\sqrt{10}}{10}
\end{array}
\right.
$$

          So we have the final equation

\begin{align*}
  \beta \frac{1}{\sqrt 2} &= \frac{\sqrt 5}{5} - \frac{i\sqrt{5}}{10} \\
  \beta &= \frac{\sqrt{10}}{5} - \frac{i\sqrt{10}}{10}
\end{align*}

    Finally we have $\ket{\varphi} =
                      (\frac{\sqrt{10}}{5} + \frac{i\sqrt{10}}{10}) \ket{\Psi^+}
                    + (\frac{\sqrt{10}}{5} - \frac{i\sqrt{10}}{10}) \ket{\Psi^-}$

        \item Compute $\langle \varphi | \Psi^+ \rangle$

$$
  \langle \varphi | \Psi^+ \rangle = \frac{2}{\sqrt 5} \times \frac{1}{\sqrt 2}
          + \frac{-i}{\sqrt 5} \times \frac{1}{\sqrt 2} =
          \frac{2}{\sqrt{10}} - \frac{i}{\sqrt{10}}
$$
      \end{enumerate}
    \end{exo}

  \section{Reversible Computation}

    We calculate the circuit $C_1$ for the values $\ket{00}, \ket{01}, \ket{10}$
    and $\ket{11}$.

    \begin{align*}
      C_1 (\ket{00}) &= \ket{00} \\
      C_1 (\ket{01}) &= \ket{10} \\
      C_1 (\ket{10}) &= \ket{01} \\
      C_1 (\ket{11}) &= \ket{11}
    \end{align*}

    We can formulate $C_1$ as follows :
    \begin{align*}
      C_1 : \mathcal H \otimes \mathcal H &\to \mathcal H \otimes \mathcal H \\
      \ket{x} \otimes \ket{y} &\mapstochar\to \ket y \otimes \ket x
    \end{align*}

    Informally, we can say that $C_1$ swap $\ket x$ and $\ket y$

  \section{Play with Controls}

    We calculate the circuit $C_2$ for the values $\ket{00}, \ket{01}, \ket{10}$
    and $\ket{11}$.

    \begin{align*}
      C_2 (\ket{00}) &= \ket{00} \\
      C_2 (\ket{01}) &= e^{i\theta} \ket{01} \\
      C_2 (\ket{10}) &= e^{i\theta} \ket{10} \\
      C_2 (\ket{11}) &= \ket{11}
    \end{align*}

    We can formulate $C_2$ as follows :
    \begin{align*}
      C_2 : \mathcal H \otimes \mathcal H &\to \mathcal H \otimes \mathcal H \\
      \ket{x} \otimes \ket{y} &\mapstochar\to \ket x \otimes
      (e^{i\theta}((x \oplus y) \oplus (1 \oplus x \oplus y))) \ket y
    \end{align*}

  \newpage

  \section{Another one}

    We calculate the circuit $C_3$ for $\ket{xy}$.

    \begin{align*}
      \mathcal{H} \otimes \mathcal{H} (\ket{xy}) =&
          \frac{1}{\sqrt 2} (\ket 0 + (-1)^x \ket 1) \otimes
          \frac{1}{\sqrt 2} (\ket 0 + (-1)^y \ket 1) \\
       =& \frac{1}{2} (\ket{00} + (-1)^x \ket{01} +
                      (-1)^y \ket{10} + (-1)^{x + y} \ket {11}) \\
      \text{CNOT} \to& 
          \frac{1}{2} (\ket{00} + (-1)^x \ket{11} +
                      (-1)^y \ket{10} + (-1)^{x + y} \ket {01}) \\
    \mathcal{H} \otimes \mathcal{H} \to& 
           \frac{1}{4} ((\ket{00} + \ket{01} + \ket{10} + \ket{11})\\
              &(-1)^x    (\ket{00} - \ket{01} - \ket{10} + \ket{11}) \\
              &(-1)^y    (\ket{00} + \ket{01} - \ket{10} - \ket{11}) \\
              &(-1)^{x+y}(\ket{00} - \ket{01} + \ket{10} - \ket{11})) \\
        =& \frac{1}{4} ((1 + (-1)^x + (-1)^y + (-1)^{x+y})\ket{00} \\
                      & (1 + (-1)^{x+1} + (-1)^y + (-1)^{x+y+1})\ket{01} \\
                      & (1 + (-1)^{x+1} + (-1)^{y+1} + (-1)^{x+y})\ket{10}) \\
                      & (1 + (-1)^x     + (-1)^{y+1} + (-1)^{x+y+1})\ket{11}
    \end{align*}

    So we have the following results :

    \begin{align*}
      C_3 \ket{00} &= \ket{00} \\ 
      C_3 \ket{01} &= \ket{11} \\
      C_3 \ket{10} &= \ket{10} \\ 
      C_3 \ket{11} &= \ket{01} 
    \end{align*}

    We can formulate $C_3$ as follows :

    \begin{align*}
     C_3 : \mathcal H \otimes \mathcal H &\to \mathcal H \otimes \mathcal H \\
      \ket{x} \otimes \ket{y} &\mapstochar\to \ket{x \oplus y} \otimes \ket y
    \end{align*}

    We can also define like that :

    \centering
    \begin{tikzpicture}
      \node (X) at (0, 0)    {$\ket x$};
      \node (Y) at (0, -1.5) {$\ket y$};

      \draw (0.5, 0)    -- (3.5, 0);
      \draw (0.5, -1.5) -- (3.5, -1.5);

      \fill[circle] (2, -1.5) circle (3pt);
      \draw[circle] (2, 0)    circle (6pt);
      \draw (2, 6pt) -- (2, -6pt);

      \draw (2, -6pt) -- (2, -1.5);
    \end{tikzpicture}

\end{document}
