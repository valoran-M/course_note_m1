\documentclass{article}

\usepackage[utf8]{inputenc}
\usepackage[T1]{fontenc}
% \usepackage[french]{babel}
\usepackage{amssymb}
\usepackage{ntheorem}
\usepackage{amsmath}
\usepackage{amssymb}
\usepackage[ a4paper, hmargin={3cm, 3cm}, vmargin={3cm, 3cm}]{geometry}

\usepackage{hyperref}
\hypersetup{
    colorlinks,
    citecolor=black,
    filecolor=black,
    linkcolor=blue,
    urlcolor=blue
}

\theoremstyle{plain}
\theorembodyfont{\normalfont}
\theoremseparator{~--}
\newtheorem*{define}{Définition}%[section]

\usepackage{tikz}
\usetikzlibrary{automata, positioning, arrows}
\usepackage{capt-of}

\usepackage{listings}
\definecolor{isarblue}{HTML}{006699}
\definecolor{isargreen}{HTML}{009966}
\definecolor{isarred}{HTML}{990066}
\lstdefinelanguage{isabelle}{%
    keywords=[1]{type_synonym,datatype,fun,abbreviation,definition,proof,
                 lemma,theorem,corollary,unfolding},
    keywordstyle=[1]\bfseries\color{isarblue},
    keywords=[2]{where,assumes,shows,and},
    keywordstyle=[2]\bfseries\color{isargreen},
    keywords=[3]{if,then,else,case,of,SOME,let,in,O},
    keywordstyle=[3]\color{isarblue},
    keywords=[4]{apply,done},
    keywordstyle=[4]\color{isarred}
}
\lstset{%
  language=isabelle,
  escapeinside={&}{&},
  columns=fixed,
  extendedchars,
  basewidth={0.5em,0.45em},
  basicstyle=\ttfamily,
  captionpos=b,
  mathescape,
}

\title{Rapport TP1}
\author{Valeran MAYTIE}
\date{}

\begin{document}
  \maketitle

  \section{Prise en main du logiciel Isabelle}

  \section{Encodage de Church}
    On cherche à encoder les entiers naturels en $\lambda$-calcul pur. Pour cela
    on utilise l'encodage de Church, il est construit avec une fonction qui
    prend en paramètre deux variables $f$ une fonction et $x$ une variable. Pour
    représenter un nombre $n$ on compose $n$ fois la fonction $f$ appliquée
    à $x$.

    Le $\lambda$-term ressemble à :
    $$
      \lambda f.\lambda x. f^n\;x \equiv 
      \lambda f.\lambda x.\;\underbrace{f\;(f\;\ldots\;(f}_{n\text{ fois}}\; x))
    $$

    En Isabelle nous définissons les entiers naturels de 0 à 5 comme ceci :

    \begin{figure}[thpb]
    \begin{lstlisting}
definition ZERO  where "ZERO  $\equiv$ $\lambda$f x. x "
definition ONE   where "ONE   $\equiv$ $\lambda$f x. f x"
definition TWO   where "TWO   $\equiv$ $\lambda$f x. f (f x)"
definition THREE where "THREE $\equiv$ $\lambda$f x. f (f (f x))"
definition FOUR  where "FOUR  $\equiv$ $\lambda$f x. f (f (f (f x)))"
definition FIVE  where "FIVE  $\equiv$ $\lambda$f x. f (f (f (f (f x))))"
    \end{lstlisting}
    \caption{Entier de Church de 0 à 5}
    \label{fig:number}
    \end{figure}

    On peut définir des opérations sur cet encodage.

  \begin{center}
    \begin{tabular}{r c l}
      $n + 1$      & : & $\lambda n\;f\;x.\;f (n\;f\;x)$          \\
      $n + m$      & : & $\lambda n\; m\;f\;x.\;m\; f\;(n\;f\;x)$ \\
      $n \times m$ & : & $\lambda n\; m\;f\;x.\;n\;(m\;f)\;x$     \\
      $n^m$        & : & $\lambda n\; m\;f\;x.\;m\; n$            \\
    \end{tabular}
  \end{center}

  Malheureusement les opérations prédécesseur et soustraction sont plus
  difficile à écrire.

  En Isabelle on défini uniquement l'addition (Figure-\ref{fig:add}) car ça sera
  la seule utile pour notre première preuve.

    \begin{figure}[thpb]
    \begin{lstlisting}
definition PLUS where "PLUS $\equiv$ $\lambda$n m f x. m f (n f x)"
    \end{lstlisting}
    \caption{Addition avec l'encodage de Church}
    \label{fig:add}
    \end{figure}

  Notre première démonstration consiste à prouver que $3 + 2 = 5$ en entier
  de Church. L'énoncé s'écrit comme ceci : \texttt{PLUS TWO THREE = FIVE}.
  Pour commencer, il faut dérouler toutes les définitions.
  On utilise donc la tactique \texttt{unfolding} avec comme argument la
  définition à déplier suivit de ``\textit{\_def}'' (par exemple pour la
  défintion \textit{PLUS} : \textit{PLUS\_def}).

  Le dépliage va effectuer tout les calculs possibles on aura donc :
  $$
  \texttt{PLUS TWO THREE }\to^*_\beta\;\lambda f\; x.\;f\;(f\;(f\;(f\;(f\;x))))
  $$

  Nous remarquons que \texttt{PLUS TWO THREE} se réduit en \texttt{FIVE}, il
  nous reste donc à monter que \texttt{FIVE = FIVE}.
  Nous avons vue en cours que l'égalité dans Isabelle est réflexiver
  ($\forall x, x = x$). Il suffit donc d'appliquer le théorème de réflexivité
  \textit{HOL.refl} que l'on peut trouver à l'aide de la commandel
  \texttt{find\_theorems "\_ = \_"}. On peut appliquer ce théorème en utilisant
  \texttt{apply(rule refl)}. La preuve complète de deux ligne :) se trouve ci
  dessous (Figure-\ref{fig:preuve}).

    \begin{figure}[htb]
    \begin{lstlisting}
lemma the_first : "PLUS TWO THREE = FIVE"
  unfolding PLUS_def TWO_def THREE_def FIVE_def
  apply(rule refl)
  done
    \end{lstlisting}
    \caption{Preuve que $2 + 3 = 5$ avec l'encodage de Church}
    \label{fig:preuve}
    \end{figure}

\end{document}
