\documentclass{article}

\usepackage[utf8]{inputenc}
\usepackage[T1]{fontenc}
\usepackage[french]{babel}
\usepackage{amssymb}
\usepackage{graphicx}
\usepackage{ntheorem}
\usepackage{amsmath}
\usepackage{amssymb}
\usepackage[ a4paper, hmargin={3cm, 3cm}, vmargin={3cm, 3cm}]{geometry}

\usepackage{hyperref}
\hypersetup{
    colorlinks,
    citecolor=black,
    filecolor=black,
    linkcolor=blue,
    urlcolor=blue
}

\theoremstyle{plain}
\theorembodyfont{\normalfont}
\theoremseparator{~--}
\newtheorem*{define}{Définition}%[section]

\usepackage{tikz}
\usetikzlibrary{automata, positioning, arrows}
\usepackage{capt-of}

\usepackage{listings}
\definecolor{isarblue}{HTML}{006699}
\definecolor{isargreen}{HTML}{009966}
\definecolor{isarred}{HTML}{990066}
\lstdefinelanguage{isabelle}{%
    keywords=[1]{type_synonym,datatype,fun,abbreviation,definition,proof,typ,
                 term,lemma,theorem,corollary,unfolding, inductive_set,
                 inductive},
    keywordstyle=[1]\bfseries\color{isarblue},
    keywords=[2]{where,assumes,shows,and},
    keywordstyle=[2]\bfseries\color{isargreen},
    keywords=[3]{if,then,else,case,of,SOME,let,in,O},
    keywordstyle=[3]\color{isarblue},
    keywords=[4]{apply,done},
    keywordstyle=[4]\color{isarred}
}
\lstset{%
  language=isabelle,
  escapeinside={&}{&},
  columns=fixed,
  extendedchars,
  basewidth={0.5em,0.45em},
  basicstyle=\ttfamily,
  captionpos=b,
  mathescape,
}

\title{Rapport TP3}
\author{Valeran MAYTIE}
\date{}

\begin{document}
  \maketitle

  \section{Datatypes and Simple Inductive Proofs}

  En Isabelle, on peut définir des listes inversées en s'inspirant des listes
  usuelles. Pour cela, on crée 2 constructeurs : \texttt{Nil} et \texttt{Snoc}
  (Figure-\ref{fig:list})

  \begin{figure}[thpb]
  \begin{lstlisting}
datatype 'a List =
  Nil
| Snoc "'a List" 'a
  \end{lstlisting}
  \caption{Liste inversées}
  \label{fig:list}
  \end{figure}

  On peut donc par exemple créé la liste : \texttt{Snoc (Snoc Nil (1::nat))
  2::nat} $\equiv$ \texttt{[1; 2]}

  Ensuite, on peut écrire des fonctions qui vont faire des opérations sur ces
  listes. On va écrire 3 fonctions : \texttt{filter}, \texttt{map} et
  \texttt{concat} (Figure-\ref{fig:fun}) :

  \begin{figure}[thpb]
  \begin{lstlisting}
fun filter
where
  "filter f Nil        = Nil"
| "filter f (Snoc l a) = 
    (if f a
     then  Snoc (filter f l) a
     else filter f l)"

fun map
where
  "map f Nil        = Nil"
| "map f (Snoc l a) = Snoc (map f l) (f a)"

fun concat
  where
  "concat S Nil = S"
| "concat S (Snoc l a) = Snoc (concat S l) a"
  \end{lstlisting}
  \caption{Fonction sur les listes}
  \label{fig:fun}
  \end{figure}

Une fois que les fonctions sont créées, on peut commencer à spécifier les
fonctions à l'aide de formule logique. Dans le TP, on veut prouver deux
formules :

\begin{itemize}
  \item $\textit{filter}\; p\;(\textit{filter}\;q\;S) =
         \textit{filter}\;(\lambda x. p\;x \wedge q\;x)\;S$
  \item $\textit{map}\;f\;(\textit{concat}\;R\;S) =
         \textit{concat}\;(\textit{map}\;f\;R)(\textit{map}\;f\;S)$
\end{itemize}

Pour prouver ces deux lemmes, on va utiliser les théorèmes de récurrence
créés quand on déclare un type à l'aide de \texttt{datatype}. On peut les
lister grâce à \texttt{find\_theorems name:List name:TP3 name:induct}. Après
avoir trouvé le bon théorème, on peut l'appliquer grâce à \texttt{apply
(rule\_tac List="?" in List.induc)}. Le point d'interrogation doit être
remplacé par la liste sur laquelle on veut faire la récurrence. Ici, on veut la
faire sur la liste $S$, car les fonctions font la récurrence sur celle-ci.

Enfin après avoir appliqué le théorème, il faut simplifier les expressions ce
qui fini la preuve. On peut raccourcir la preuve en utilisant la tactique
\texttt{induct\_tac "S"}.

\section{Inductive sets - Inductive Proofs}

Dans cette section, on cherche à définir un prédicat inductif pour définir les
ensembles contenant uniquement les nombre pairs (Figure-\ref{fig:ev}).

    \begin{figure}[thpb]
  \begin{lstlisting}
inductive_set Even :: "int set"
where
  a: "0 $\in$ Even"
| b: "n $\in$ Even $\Rightarrow$ (n + 2) $\in$ Even"
| c: "n $\in$ Even $\Rightarrow$ (n - 2) $\in$ Even"
  \end{lstlisting}
  \caption{Ensemble des nombres pairs}
  \label{fig:ev}
  \end{figure}

  Les notations $a$, $b$ et $c$ permettent de donner des noms aux lemmes
  associés. On peut donc maintenant prouver assez facilement que
  $2 \in \textit{Even}$ (Figure-\ref{fig:2ev}.

  \begin{figure}[thpb]
  \begin{lstlisting}
lemma "2 $\in$ Even"
  using Even.a Even.b by force
  \end{lstlisting}
  \caption{Ensemble des nombres pairs}
  \label{fig:2ev}
  \end{figure}

  Malheureusement, montrer que $1 \not \in \textit{Even}$ est bien plus
  difficile. Pour faire cella, il faut d'abord montrer :
  $x \in \textit{Even} \Rightarrow \exists k.\; x = 2 \times k$ et ensuite, on
  peut appliquer ce lemme et montrer qu'il n'existe pas de $k$ telle que $1 = 2
  \times k$.

\section{Modeling Exercise}

Dans cette partie, on va chercher à modéliser le $\lambda$-calcul dans Isabelle.
Pour cella, on commence à définir le type des $\lambda$-termes
(Figure-\ref{fig:lambda})

  \begin{figure}[thpb]
  \begin{lstlisting}
                datatype const =                    datatype terms =
                    Nat nat                           Var string
                  | Bool bool                       | Const const
                                                    | App terms terms
                                                    | Lambda string terms
  \end{lstlisting}
  \caption{$\lambda$-termes avec constante}
  \label{fig:lambda}
  \end{figure}

\newpage
Ensuite, on veut modéliser la règle de typage vue dans le premier cours. Donc,
on définit les types possibles (Figure-\ref{fig:types})

  \begin{figure}[thpb]
  \begin{lstlisting}
datatype types =
  TVar string
| TNat
| TBool
| Arrow types types (infixr "$\to$" 200)
  \end{lstlisting}
  \caption{Définitino des types}
  \label{fig:types}
  \end{figure}

On peut définir des notations assez simplement (plus facilement quand Coq) à
l'aide du mot clé \texttt{infixr/l}. Le type fonction sera donc écrit avec une
flèche dans les spécifications.

Maintenant, il faut définir la fonction \texttt{instantiate}. Elle va remplacer
un type variable par un autre type. Elle s'écrit assez simplement par récurrence
sur les types (Figure-\ref{fig:inst}).

  \begin{figure}[thpb]
  \begin{lstlisting}
fun instantiate :: "types $\Rightarrow$ string $\Rightarrow$ types $\Rightarrow$ types"
where
  "instantiate (t1 $\to$ t2) s t = (instantiate t1 s t) $\to$ (instantiate t2 s t)"
| "instantiate TNat s t = TNat"
| "instantiate TBool s t = TBool"
| "instantiate (TVar vs) s t = (if vs = s then t else (TVar vs))"
  \end{lstlisting}
  \caption{Fonction \texttt{instantiate}}
  \label{fig:inst}
  \end{figure}

Enfin, on peut modéliser les règle d'inférence vue dans le premier cours à
l'aide d'un prédicat inductif (Figure-\ref{fig:ptype}) 

  \begin{figure}[thpb]
  \begin{lstlisting}
inductive typing :: "(const $\rightharpoonup$ types) $\Rightarrow$ (string $\rightharpoonup$ types) $\Rightarrow$ terms $\Rightarrow$ types $\Rightarrow$ bool"
    ("\_, \_ $\vdash$ \_ : \_" [50, 50, 50] 50)
  where
    Var   : "$\Gamma$ x = Some T $\Rightarrow$ $\Sigma$, $\Gamma$ $\vdash$ Var x : T"
  | Const : "$\Sigma$ c = Some C $\Rightarrow$ $\Sigma$, $\Gamma$ $\vdash$ Const c : instantiate C x T"
  | Abs   : "$\Sigma$, $\Gamma$ (v $\mapsto$ T) $\vdash$ t : U $\Rightarrow$ $\Sigma$, $\Gamma$ $\vdash$ Abs v t : (T $\to$ U)"
  | App   : "$\Sigma$, $\Gamma$ $\vdash$ t1 : (T2 $\to$ T1) $\Rightarrow$ $\Sigma$, $\Gamma$ $\vdash$ t2 : T2 $\Rightarrow$ $\Sigma$, $\Gamma$ $\vdash$ (App t1 t2) : T1"
  \end{lstlisting}
  \caption{Fonction \texttt{instantiate}}
  \label{fig:ptype}
  \end{figure}

  Pour représenter les environnements, on utilise les \texttt{Map} de Isabelle
  qui sont représenter dans le type par une fonction partiel
  ($\rightharpoonup$).

  Pour rendre les règles plus lisibles, on va défini la même notation qui est
  utilisée sur papier ce qui donne un prédicat inductif assez joli et visuel.

\end{document}
